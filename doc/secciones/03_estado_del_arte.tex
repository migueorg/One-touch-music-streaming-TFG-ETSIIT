\chapter{Estado del arte}

El software libre y sus licencias \cite{gplv3} ha permitido llevar a cabo una expansión del
aprendizaje de la informática sin precedentes.

\section{Análisis de las posibles soluciones al problema propuesto}

Si intentamos solucionar el problema introducido al principio del documento
usando las alternativas existentes actualmente nos vamos a encontrar con
soluciones poco amigables con el usuario, poco eficaces, soluciones de código
privativo y/o demasiado caras.

Usaré el siguiente ejemplo práctico para testear las soluciones actuales y
evaluar el estado del arte: Si llegamos a casa del trabajo, y hemos vuelto
escuchando un podcast en nuestros auriculares inalámbricos mientras andábamos;
al llegar a casa y ponernos a preparar la comida nos gustaría poder seguir
escuchando el podcast por el mismo minuto en el que íbamos pero en los altavoces
de casa. 

\subsection{Solución tradicional}
Para hacer esto, simplemente podríamos conectar el móvil al altavoz mediante el
\emph{jack} de 3.5, sin embargo, los móviles actualmente carecen de dicho puerto
(salvo pocas excepciones). 

\subsection{Solución asistente de voz}
Otra opción sería emplear cualquiera de los asistentes de voz como lo son
(\emph{Alexa, Google Assistant, Siri}) y pedirle que reproduzca el podcast que
estábamos escuchando, el problema de esto es que tendremos que recordar el
nombre concreto del podcast, tenemos que ponernos a hablarle al asistente lo
cual rompe la sensación de integración y continuidad, y este tiene que ser capaz
de entendernos correctamente, lo cual no pasa siempre. Adicionalmente,
requeriríamos de un altavoz inteligente con micrófono activo de cualquiera de las
marcas mencionadas anteriormente, el cual debería incorporar dicho asistente de
voz. 

\subsection{Solución \emph{big tech}}
Otra opción más, la cual es compatible con la mayoría de altavoces con
asistentes integrados, sería la de tener el altavoz en red y mandar el podcast
desde el móvil a través de protocolos como \emph{Google Cast, AirPlay o Spotify
Connect}, lo cual es una de las opciones más rápidas y sencillas, pero implica
tener que estar navegando a través de aplicaciones e interfaces hasta llegar a
la opción de enviar contenido. Y en el caso de que tengamos muchos altavoces en
red, tendremos que elegir de entre una lista amplia de dispositivos, al cual
queremos enviar dicho contenido. De nuevo, este último caso, aún siendo el más
factible, rompe mucho con la sensación de integración y continuidad de la que
hablé anteriormente. 

\subsection{Solución \emph{Apple}} Sin embargo, \emph{Apple} sí que tiene una
solución para esto integrada en su ecosistema. Para los usuarios que tengan un
\emph{iPhone} y un \emph{HomePod} \cite{HomePod}, permiten que la reproducción
del \emph{iPhone} continúe en el \emph{HomePod} simplemente con acercar el
\emph{iPhone} al \emph{HomePod}, sin la necesidad de navegar por interfaces ni
seleccionar nada. Todo de forma transparente para el usuario y en cuestión de
segundos. 

Para ello \emph{Apple} emplea un chip propio llamado \emph{Apple U1} \cite{U1}
en ambos dispositivos (\emph{iPhone} y \emph{HomePod}), el cual es un chip que
dota a los dispositivos de conectividad a través de
\href{https://en.wikipedia.org/wiki/Ultra-wideband}{\emph{UWB (Ultra
Wideband)}}. Este chip trabaja en una radiofrecuencia entre los 6 y 8GHz,
permitiendo así una comunicación de bajo alcance, muy bajo consumo, pero alto
ancho de banda a partir de los estándares
\href{https://es.wikipedia.org/wiki/IEEE_802.15#Grupo_de_trabajo_4_(WPAN_de_baja_velocidad)}{IEEE
802.15.4a y IEEE 802.15.4z}, y es usado para determinar una precisión espacial
entre los dispositivos que la porten. Gracias a esto, tanto \emph{iPhone} como
\emph{HomePod} saben que están apuntándose el uno al otro, para, acto seguido
mediante \emph{Apple Handoff} \cite{Handoff} comunicarse a través de
\emph{bluetooth} y \emph{wifi}, y de esta manera conocer la canción que se estaba
reproduciendo en el \emph{iPhone} y compartir la reproducción a través de
\emph{AirPlay} \cite{AirPlay}.

El problema de esta solución es que no es de código abierto, en la que además se emplean
tecnologías propias inaccesibles para cualquier usuario ajeno al ecosistema de Apple.

\section{Conclusión tras estudiar el mercado}
Tras repasar todas las posibilidades existentes para solventar el problema
actualmente, vemos que la solución más elegante es la que tiene \emph{Apple}. El
problema es que esa solución es de código privativo, además de estar en un
ecosistema muy cerrado, siendo exclusiva de algunos dispositivos \emph{Apple}.
Si queremos disfrutar de la automatización y transparencia que se aprecia en
esta solución no encontramos nada similar ni cercano. Es por eso que nace la
necesidad de crear este proyecto, persiguiendo un resultado similar al que
proporciona \emph{Apple}, pero enfocado en dispositivos \emph{Android} con
tecnologías más comunes y baratas.
