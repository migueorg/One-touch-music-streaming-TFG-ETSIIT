\chapter{Estado del arte}

El software libre y sus licencias \cite{gplv3} ha permitido llevar a cabo una expansión del
aprendizaje de la informática sin precedentes.

Si intentamos solucionar el problema introducido al principio del documento
usando las alternativas existentes actualmente nos vamos a encontrar con
soluciones poco amigables con el usuario, poco eficaces, soluciones de código
privativo y/o demasiado caras.

Usaré el siguiente ejemplo práctico para testear las soluciones actuales y
evaluar el estado del arte: Si llegamos a casa del trabajo, y hemos vuelto
escuchando un podcast en nuestros auriculares inalámbricos mientras andábamos;
al llegar a casa y ponernos a preparar la comida nos gustaría poder seguir
escuchando el podcast por el mismo minuto en el que íbamos pero en los altavoces
de casa. 

Para hacer esto, actualmente podríamos conectar el móvil al altavoz mediante el
\emph{jack} de 3.5, sin embargo, los móviles actualmente carecen de dicho puerto
(salvo pocas excepciones). 

Otra opción sería emplear un asistente virtual (\emph{Alexa, Google Assistant,
Siri}) y pedirle que reproduzca el podcast que estábamos escuchando, el problema
de esto es que tendremos que recordar el nombre concreto del podcast, tenemos
que ponernos a hablarle al asistente lo cual rompe la sensación de integración y
continuidad, y a todo esto el asistente tiene que ser capaz de entendernos
correctamente, lo cual no pasa siempre. 

Otra opción más, la cual es compatible con la mayoría de altavoces con
asistentes integrados, sería la de tener el altavoz en red y mandar el podcast
desde el móvil a través de protocolos como \emph{Google Cast, AirPlay o Spotify
Connect}, lo cual es una de las opciones más rápidas y sencillas, pero implica
tener que estar navegando a través de aplicaciones e interfaces hasta llegar a
la opción de enviar contenido. Y en el caso de que tengamos muchos altavoces en
red, tendremos que elegir de entre una lista amplia de dispositivos, al cual
queremos enviar dicho contenido. De nuevo, este último caso, aún siendo el más
factible, rompe mucho con la sensación de integración y continuidad de la que
hablé anteriormente. 

Sin embargo, \emph{Apple} sí que tiene una solución para esto integrada en su
ecosistema. Para los usuarios que tengan un \emph{iPhone} y un \emph{HomePod},
permiten que la reproducción del \emph{iPhone} continúe en el \emph{HomePod}
simplemente con acercar el \emph{iPhone} al \emph{HomePod}, sin la necesidad de
navegar por interfaces ni seleccionar nada. Todo de forma transparente para el
usuario y en cuestión de segundos. El problema de esto es que fuera de ese
ecosistema cerrado de \emph{Apple}, no hay ninguna opción más. Si un usuario
quiere disfrutar de esa transparencia y simplicidad, y no tiene \emph{iPhone},
bien por el precio del mismo o bien porque prefiere tener un sistema más abierto
como es \emph{Android}, no le será posible disfrutar de esa opción que mantenga
la sensación de integración y continuidad. A su vez, esta opción limita la
posibilidad de escucha a únicamente el altavoz del \emph{HomePod} o de los
dispositivos \emph{HomePods} que haya emparejados en casa. Por lo que en caso de
tener un equipo propio de alta fidelidad o ajeno al ecosistema de \emph{Apple},
tampoco se podrá usar.\\

