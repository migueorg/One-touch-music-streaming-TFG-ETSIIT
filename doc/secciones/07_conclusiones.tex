\chapter{Conclusiones y trabajos futuros}

\section{Conclusiones}

Finalmente, se ha conseguido llevar a cabo la solución propuesta, cumpliendo los
objetivos que establecimos al principio en la
\hyperref[introduccion]{introducción del proyecto}, los cuales han ido guiando
el proyecto correctamente desde el inicio. Así pues, si sometemos a juicio el
resultado final, con los objetivos iniciales:

\begin{itemize}
    \item \textit{Se deberá poder transmitir o mandar el sonido que se esté
    reproduciendo en el dispositivo móvil hasta el dispositivo que haga de
    altavoz o receptor}: \textbf{Se satisface el objetivo pues el envío es del
    objeto con la captura de todo el audio interno del Android}
    \item \textit{La comunicación entre ambos dispositivos (móvil y altavoz)
    deberá de llevarse a cabo en menos de 3 segundos para poder mantener la
    sensación de continuidad de la que hemos hablado antes. Más de 3 segundos
    puede hacer creer al usuario que el servicio no está funcionando}: \textbf{Se
    satisface el objetivo pues al emplear un protocolo que no es orientado a
    conexión, la transferencia es inmediata.}
    \item \textit{Facilidad de uso y sencillez a la hora de compartir el
    contenido del reproductor al altavoz. El proceso de envío de sonido desde el
    Android hasta el altavoz ha de poder realizarse sin interactuar con menús}:
    \textbf{Se satisface el objetivo pues la única interacción es la de acercar
    el dispositivo al altavoz y aceptar la captura de audio, siendo esto último
    una medida de seguridad de Android.}
\end{itemize}

Por lo que podemos afirmar que con la realización de este proyecto se ha
conseguido aportar una solución más al estado del arte, teniendo en cuenta las
necesidades de los usuarios. 

Algunas conclusiones que saco tras haber investigado la informática a bajo
nivel, tanto para comprender el funcionamiento del NFC, como el del audio y el
de los paquetes de red; es que es bastante más complicado de lo que parece desde
fuera, ya que al estar acostumbrados a trabajar a un nivel de abstracción más
superior gracias a las API, bibliotecas y frameworks, parece que olvidamos el
nivel de complejidad que tiene una simple llamada a una función en una
biblioteca. Como ventaja, el programar y trabajar a tan bajo nivel hace que las
soluciones programando a bajo nivel sean más sofisticadas, y por lo general, si
se implementa bien, puedan llegar a tener un mejor rendimiento.

También me gustaría destacar, que a día de hoy da la impresión de que el NFC es
una tecnología olvidada sobre la cual las empresas tecnológicas estiman que no
hay más que exprimir más allá de su uso principal hoy en día para los pagos, sin
embargo, y como hemos visto en este proyecto, pueden existir aplicaciones y usos
muy interesantes. Sin olvidar, su coste tan reducido y la escasa necesidad de
hardware para implementarlo.

A nivel personal, creo que realizar este proyecto de ingeniería desde 0,
siguiendo metodologías ágiles, me ha aportado unas muy buenas prácticas a la
hora de afrontar cualquier proyecto, y una soltura a la hora de trabajar y
desenvolverme con técnicas, herramientas y flujos de trabajo que se utilizan a
día de hoy en los entornos reales de trabajo. 

\section{Trabajos futuros}

Como trabajos futuros pienso que hay bastante donde seguir experimentando y
expandiendo el proyecto. Como por ejemplo:

\begin{itemize}
    \item Escribir los test creando mocks que verifiquen el comportamiento de
    los dispositivos hardware involucrados.
    \item Añadir un \emph{workflow} automatizado que ejecute los test cuando
    detecte alguna modificación en el código.
    \item Añadir un botón en la IU de la aplicación para detener correctamente
    el streaming.
    \item Añadir un \emph{linter} de \emph{Kotlin} en forma de \emph{GitHub
    Action} que compile el código cada vez que se modifique.
    \item Implementar una codificación de audio para emplear un protocolo
    destinado a audio en streaming y comparar latencias frente a calidad de
    audio y consumo de recursos.
    \item Implementar el envío de la dirección IP e inicio de streaming usando
    \emph{BLE} y comparar frente al \emph{NFC} en un entorno real. 
\end{itemize}

Definitivamente, todo esto es trabajo que tengo intención de realizar en el
futuro, pues este proyecto era una idea que me entusiasmaba desde el primer
momento que lo presenté, y la cual me gustaría seguir desarrollando e impulsando
por mi cuenta.