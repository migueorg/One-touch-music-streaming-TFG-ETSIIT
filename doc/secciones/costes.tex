\chapter{Costes}
En este capítulo se analizarán los costes de desarrollo del proyecto, teniendo
en cuenta las inversiones que han sido necesarias para poder realizarlo. Así
como el de producto, teniendo en cuenta el coste de implementar el
proyecto por un usuario externo una vez desarrollado.

\section{Costes de desarrollo}

\subsection{Hardware}

Para poder llevar a cabo este proyecto se ha necesitado el siguiente hardware,
con el respectivo coste:

\subsubsection{Costes de equipo de desarrollo}

Atendiendo al cálculo de amortización de equipos informáticos
\cite{costes-explicacion}, y siguiendo los criterios establecidos por el marco
legal decretado en el BOE \cite{costes-boe}, aplicando una amortización lineal.

\begin{table}[H]
    \begin{center}
    \begin{tabular}{| l | c | c | c |}
        \hline
        \textbf{Hardware} & \textbf{Precio} & \textbf{Amortización} & \textbf{Coste} \\ \hline
        MacBook Pro & 2.029,00\euro & 12,5\% en 8 años. & 253,63\euro/año.\\
        Monitor LG & 105,48\euro & 12,5\% en 8 años. & 13,19\euro/año.\\
        OnePlus 6 & 503,53\euro & 25\% en 4 años. & 125,89\euro/año.\\
        Router Asus & 21,99\euro & 12,5\% en 8 años. & 2,75\euro/año.\\ \hline
        \textbf{Coste equipo completo}: & & & 395,46\euro/año \\ \hline
        \textbf{Tiempo de desarrollo}: & & 9 meses (2 años fiscales diferentes) & \textbf{Total: 790,92\euro} \\ \hline
    \end{tabular}
    \caption{Amortización de los costes.}
    \label{tab:costes-hardware-desarrollo}
    \end{center}
\end{table} 

\subsection{Software}
Para desarrollar este proyecto no ha sido necesario ningún software ni
biblioteca de pago.

\section{Costes de producto}

Para una persona que desea implementar este proyecto una vez desarrollado, como
mínimo (se especificarán las alternativas más baratas que se encuentren) tendría
el siguiente coste:

\subsection{Opción \emph{Plug \& Play}}

Para esta configuración lo más barata y a la vez simple posible se empleará un
portátil. \footnote{Se ha elegido un portátil porque de esta forma no hay que
añadir teclado, ratón, pantalla y altavoz a la configuración.}
\begin{table}[H]
    \begin{center}
    \begin{tabular}{| l | c | c | l |}
        \hline
        \textbf{Hardware} & \textbf{Precio} & \textbf{Link compra} & \textbf{Observaciones} \\ \hline
        
        Android con versión 10 y NFC & 84,99\euro & \href{https://www.amazon.es/CUBOT-Tel%C3%A9fono-Smartphone-Expandir-Octa-Core/dp/B0CC238QJG/ref=sr_1_6?__mk_es_ES=%C3%85M%C3%85%C5%BD%C3%95%C3%91&crid=2BYDWM54C15NI&keywords=android%2Bnfc&qid=1699844074&sprefix=android%2Bnfc%2Caps%2C137&sr=8-6&th=1}{Amazon}& CUBOT P50 \\ 
        Portátil con Linux & 189\euro & \href{https://www.amazon.es/ASUS-Chromebook-CX1500CNA-BR0110-Ordenador-operativo/dp/B0BT537K12/ref=sr_1_11?crid=28A51ZD20XKEI&keywords=portatil&qid=1699843810&sprefix=portati%2Caps%2C365&sr=8-11&th=1}{Amazon} & ASUS CX1500CNA\\
        Módulo NFC USB & 14,99\euro & \href{https://www.amazon.es/Digitalkey-PN532-Mdoulo-Lectura-Escritura/dp/B07ZWQ7Q32/ref=sr_1_2?crid=1M0R0S3UON8X3&keywords=pn532+usb&qid=1699844387&sprefix=pn532+%2Caps%2C107&sr=8-2}{Amazon} & PN532 a USB\\ \hline

        \textbf{Total:} & 288,98\euro \\ 
        \cline{1-2}
    \end{tabular}
    \caption{Coste producto \emph{Plug \& Play}.}
    \label{tab:costes-product}
    \end{center}
\end{table} 

\subsection{Opción \emph{DIY (Do It Yourself)}}
Para esta configuración se empleará una Raspberry Pi en lugar de un portátil.


\begin{table}[H]
    \begin{center}
    \begin{tabular}{| l | c | c | l |}
        \hline
        \textbf{Hardware} & \textbf{Precio} & \textbf{Link compra} & \textbf{Observaciones} \\ \hline
        Android 10 y NFC & 84,99\euro & \href{https://www.amazon.es/CUBOT-Tel%C3%A9fono-Smartphone-Expandir-Octa-Core/dp/B0CC238QJG/ref=sr_1_6?__mk_es_ES=%C3%85M%C3%85%C5%BD%C3%95%C3%91&crid=2BYDWM54C15NI&keywords=android%2Bnfc&qid=1699844074&sprefix=android%2Bnfc%2Caps%2C137&sr=8-6&th=1}{Amazon}& CUBOT P50 \\ 
        Raspberry Pi 1b & 45,78\euro & \href{https://es.aliexpress.com/item/1005004697682601.html?spm=a2g0o.productlist.main.15.33eb6bfd05rMkD&algo_pvid=07c626ee-31e1-410d-bcb8-42ab1a6963f9&algo_exp_id=07c626ee-31e1-410d-bcb8-42ab1a6963f9-7&pdp_npi=4%40dis%21EUR%2150.87%2145.78%21%21%2153.02%21%21%4021059dbe16998414418033546e3554%2112000030138165340%21sea%21ES%210%21AB&curPageLogUid=0G4zog4X6MqN}{AliExpress} & \\ 
        PN532  & 5,68\euro & \href{https://es.aliexpress.com/item/4001169120990.html?spm=a2g0o.productlist.main.3.1373a970Bq2vId&algo_pvid=a03541c9-012e-4322-8a96-1d616775bc64&algo_exp_id=a03541c9-012e-4322-8a96-1d616775bc64-1&pdp_npi=4%40dis%21EUR%214.87%214.38%21%21%215.08%21%21%402103205216998406149955402ee3ba%2110000015015464836%21sea%21ES%210%21AB&curPageLogUid=L7L6DK7DBqn7}{AliExpress} & Módulo NFC.\\
        Cables jumper H-H & 3,47\euro & \href{https://es.aliexpress.com/item/32825558073.html?spm=a2g0o.productlist.main.3.77ef6976H8wvFc&algo_pvid=8816e97a-1a7e-4ce4-8444-8640a432a291&algo_exp_id=8816e97a-1a7e-4ce4-8444-8640a432a291-1&pdp_npi=4%40dis%21EUR%210.92%210.67%21%21%210.96%21%21%40210313e916998409290625278e091f%2112000036201923577%21sea%21ES%210%21AB&curPageLogUid=HLzROKKGNukF}{AliExpress} & Conectan el PN532 con la Raspberry Pi.\\
        Cable ethernet 1m & 2,02\euro & \href{https://www.amazon.es/NANOCABLE-10-20-0405-Cable-Ethernet-latiguillo/dp/B00AKBSB1E/ref=sr_1_11?crid=1VAW2QPU0GXCA&keywords=cable%2Bethernet&qid=1699842769&sprefix=cable%2Bet%2Caps%2C538&sr=8-11&th=1}{Amazon} & Conectan la Raspberry Pi con el router.\\
        Alimentador 5V & 13,50\euro & \href{https://www.kubii.com/es/fuentes-de-alimentacion/3456-fuente-de-alimentacion-raspberry-pi-micro-usb-125w-3272496308534.html#/337-version_d_alimentation-union_europea_u_e_?src=raspberrypi}{Kubii} & Adaptador oficial.\\
        Micro SD 32GB & 11,37\euro & \href{https://www.amazon.es/SanDisk-Extreme-microSDHC-Adaptador-Velocidad/dp/B06XWMQ81P/ref=sr_1_5?qid=1699841213&refinements=p_n_feature_browse-bin%3A948154031&s=computers&sr=1-5}{Amazon} & Para el SO de la Raspberry Pi.\\
        Altavoz jack 3.5 & 4,89\euro & \href{https://www.amazon.es/Tacens-Anima-AS1-Altavoces-alimentaci%C3%B3n/dp/B00II0QHX8/ref=sr_1_14?crid=PJIDP8E6RSZL&keywords=altavoces%2Bpc&qid=1699843171&sprefix=alta%2Caps%2C161&sr=8-14&th=1}{Amazon} & Para conectar a la Raspberry Pi \\
        \hline
        \textbf{Total:} & 171,7\euro \\ 
        \cline{1-2}
    \end{tabular}
    \caption{Coste producto \emph{DIY}.}
    \label{tab:costes-diy}
    \end{center}
\end{table} 




Estos precios serían los mínimos para poder utilizar el proyecto, sin embargo,
si ya se posee alguno de los componentes necesarios se podría omitir su compra,
reduciendo de esta manera los costes.