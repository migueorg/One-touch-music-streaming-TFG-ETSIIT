\chapter{Introducción}

Este proyecto es software libre, y está liberado con la licencia \cite{gplv3}.\\

Puede encontrarse en el siguiente repositorio:
\url{https://github.com/migueorg/One-touch-music-streaming-TFG-ETSIIT/}, en el
cual se ha desarrollado desde el principio en abierto.

\section{Descripción del problema}
En los tiempos en los que estamos, cada vez es más normal escuchar música o
podcasts de fondo mientras hacemos otras tareas de nuestro día a día. Escuchamos
contenido mientras trabajamos, vamos en el coche, de camino al trabajo, de
vuelta a casa, mientras cocinamos, limpiamos, hay quienes también escuchan algún
tipo de podcast, ruido blanco o sonido para irse a dormir, entre muchos otros
más escenarios. 

A su vez, en el contexto informático en el que estamos, también le hemos dado la
espalda a los cables siempre que nos sea posible. Los auriculares inalámbricos
son un obligatorio en el día a día de muchas personas, facilitando así la
integración de la música en cada vez más contextos diferentes de nuestra vida.
Sin embargo, aún teniendo en cuenta los avances en el campo del \emph{IoT}, no
es tan fácil mantener esa sensación de integración y continuidad respecto a la
escucha de música. Ya que si queremos seguir escuchando en algún altavóz
inteligente la misma canción o podcast que veníamos escuchando en el móvil,
tendremos que empezar a interactuar con menús, dictar comandos (que no siempre
fucnionan) al asistente virtual; o alguna otra alternativa compleja o cara.\\

\section{Motivación}
Por tanto, una vez puestos en contexto, la motivación de este proyecto es la de
poder brindar dicha sensación de continuidad al sistema operativo \emph{Android}
con un coste notablemente más bajo que el de la solución de \emph{Apple}. Y es
que esa sensación de la que tanto hincapié se ha hecho a lo largo de la
descripción, no es más que una manera de decir facilidad de utilización,
simpleza, accesibilidad y funcionamiento automático e independiente. Los cuales
van a ser parte de los objetivos principales que guíen este proyecto (más
adelante se hablará más en concreto de los objetivos). Por otro lado, además de
poder solucionar dicho problema en el ecosistema de \emph{Android}, con este
proyecto también se va a buscar la posibilidad de emplear altavoces que no sean
\emph{HomePod}. 

\section{Objetivos}
\begin{itemize}
    \item Se deberá poder transmitir o mandar el sonido que se esté
    reproduciendo en el dispositivo móvil hasta el dispositivo que haga de
    altavoz o receptor.
    \item La comunicación entre ambos dispositivos (móvil y altavoz) deberá de
    llevarse a cabo en menos de 3 segundos para poder mantener la sensación de
    continuidad de la que hemos hablado antes. Más de 3 segundos puede hacer
    creer al usuario que el servicio no está funcionando.
    \item Facilidad de uso y sencillez a la hora de compartir el contenido del
    reproductor al altavoz. El proceso de envío de sonido desde el Android hasta
    el altavoz ha de poder realizarse sin interactuar con menús.
    \item El coste final del producto debe estar por debajo de los 349€, el cual
    es el precio del HomePod a secas.
\end{itemize}

\section{Personajes}
Los personajes que tienen el problema presentado anteriormente y que, por tanto,
son los que usarán la aplicación para solventarlo son:
\begin{itemize}
    \item Pepe, 32 años, es un contable al que le gusta la domótica, el
    \emph{DIY}; y se considera bastante \emph{geek}. Tiene su casa llena de
    automatizaciones, pero encuentra los altavoces inteligentes muy poco
    interesantes, no los considera inteligentes porque siempre tiene que decirle
    lo que quiere hacer de forma muy mascada para que le consigan entender
    correctamente. 
    \item María, 55 años, es audiófila desde pequeña y siempre le ha gustado
    escuchar música para evadirse del mundo y sus problemas. Tras llevar media
    vida en el mundo del audio, tiene muchos equipos que, a pesar de funcionar
    como el primer día, no puede utilizarlos con plataformas actuales. Le gustaría
    darles una segunda vida simplificando su funcionamiento e interacción con
    ellos lo más posible. 
    \item El piso de Olga es un grupo de jóvenes de entre 20 y 25 años que están
    acostumbrados a interactuar con la tecnología, sin embargo, quieren unificar el control y la
    interacción de la reproducción de música para todos los integrantes del
    grupo en un mismo dispositivo de manera simple y sin tener que prestarse los
    móviles.
\end{itemize}

\section{User journeys}
Los contextos en los que los personajes citados arriba usarán la aplicación son
los siguientes:\\

\begin{itemize}
    \item Pepe, vive a 20 minutos andando del trabajo, por lo que va a
    pie para ahorrar gasolina y mantenerse activo. Al salir del trabajo siempre
    se pone un podcast que dura 30 minutos para hacer su trayecto más ameno, el
    cual lo va escuchando en sus auriculares inalámbricos. Cuando llega a casa
    se prepara algo de comer mientras escucha los últimos minutos de su podcast.
    Para ello solo tendrá que acercar su Android hasta el altavoz en el que
    desea escucharlo.
    \item María dedica la tarde de los domingos a escuchar música en su casa
    usando su equipo convencional que tiene desde hace 20 años (cuando
    fabricaban cosas de calidad); debido a la antigüedad este carece de conexión
    a internet. Así que para poder disfrutar de música actual sin tener que
    conectar el móvil al equipo de música a través de un cable, tiene conectado
    su antiguo ordenador con un módulo lector NFC, así no tendrá más que acercar el
    móvil con la canción que desea a dicho lector para empezar a disfrutar de su
    equipo \emph{vintage}.
    \item En el piso de Olga organizan fiestas esporádicamente, por lo que la
    música es un tema principal que estará involucrado durante toda la noche hasta que
    acabe el evento. A lo largo de la fiesta cada asistente va a querer poner
    canciones que le interesen, para ello simplemente tendrá que acercar su
    Android con la canción deseada para que esta se añada a la cola del altavoz.
\end{itemize}


\section{Historias de usuario}
A partir de los personajes y los user story presentados anteriormente, podemos
extraer ciertas funcionalidades del proyecto que el público destino espera
obtener o poder hacer. Estas funcionalidades clave tienen forma de las
siguientes historias de usuario:

\begin{enumerate}
    \item \href{https://github.com/migueorg/One-touch-music-streaming-TFG-ETSIIT/issues/10}{HU1}: Como usuario que busca la mayor automatización y transparencia
    con la tecnología posible (Pepe) quiero poder seguir escuchando mi
    reproducción actual manteniendo el mismo momento de escucha sin tener que
    interactuar con ningún menú al enviar hacia el altavoz.
    \item \href{https://github.com/migueorg/One-touch-music-streaming-TFG-ETSIIT/issues/11}{HU2}: Como usuario con equipo de audio propio (María) quiero reutilizar
    mis propios altavoces y equipos viejos con una inversión baja.
    \item \href{https://github.com/migueorg/One-touch-music-streaming-TFG-ETSIIT/issues/13}{HU3}: Como grupo de usuarios que empleará diferentes móviles en un solo
    altavoz (Piso Olga) queremos poder acceder todos al mismo altavoz o receptor
    sin tener que configurar nada cada vez que se va a usar un móvil nuevo.
    \item \href{https://github.com/migueorg/One-touch-music-streaming-TFG-ETSIIT/issues/14}{HU4}: Como persona ajena al proyecto quiero conocer las decisiones que
    se han tomado al desarrollar el proyecto para poder entender por qué se han
    empleado ciertas plataformas y componentes y poder continuar o modificar su
    desarrollo si fuese necesario.
\end{enumerate}

Nótese que además de las funcionalidades que el usuario final espera obtener,
también se ha dedicado una historia de usuario (HU5) que personifica al tribunal o
a cualquier persona que en el futuro encuentre este proyecto y quiera conocer
por qué se ha actuado de la forma que se ha hecho durante todo el trabajo.

\section{Milestones}
Los productos mínimamente viables o \emph{releases} que se irán haciendo durante el
desarrollo se han estimado que serán los siguientes:

\begin{enumerate}
    \item \href{https://github.com/migueorg/One-touch-music-streaming-TFG-ETSIIT/milestone/1}{M0}: Repositorio y documentación funcional sobre la que empezar a programar.
    \item \href{https://github.com/migueorg/One-touch-music-streaming-TFG-ETSIIT/milestone/2}{M1}: Captar el audio del teléfono Android para transmitirlo.
    \item \href{https://github.com/migueorg/One-touch-music-streaming-TFG-ETSIIT/milestone/3}{M2}: Recibir en un dispositivo diferente el audio previamente transmitido.
    \item \href{https://github.com/migueorg/One-touch-music-streaming-TFG-ETSIIT/milestone/4}{M3}: Automatizar la tarea de envío y recepción en un solo gesto/toque
    \item \href{https://github.com/migueorg/One-touch-music-streaming-TFG-ETSIIT/milestone/5}{M4}: Memoria entregable para el tribunal.
    \item \href{https://github.com/migueorg/One-touch-music-streaming-TFG-ETSIIT/milestone/6}{M5}: Presentación.
\end{enumerate}