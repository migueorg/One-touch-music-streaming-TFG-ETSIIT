\chapter{Introducción}

Este proyecto es software libre, y está liberado con la licencia \cite{gplv3}.\\

Puede encontrarse en el siguiente repositorio: \url{https://github.com/migueorg/One-touch-music-streaming-TFG-ETSIIT/}

\section{Descripción del Problema}
En los tiempos en los que estamos, cada vez es más normal escuchar música o podcast de fondo mientras hacemos otras tareas de nuestro día a día. Escuchamos contenido mientras trabajamos, vamos en el coche, de camino al trabajo, de vuelta a casa, mientras cocinamos, limpiamos, hay quienes también escuchan algún tipo de podcast, ruido blanco o sonido para irse a dormir, entre muchos otros más escenarios. Y a su vez, en el contexto informático en el que estamos, también le hemos dado la espalda a los cables siempre que nos sea posible. Los auriculares inalámbricos son un obligatorio en el día a día de muchas personas, facilitando así la integración de la música en cada vez más contextos diferentes de nuestra vida. Sin embargo, aún teniendo en cuenta los avances en el campo del IoT, no es tan fácil mantener esa sensación de integración y continuidad respecto a la escucha de música.\\

Por ejemplo si llegamos a casa del trabajo, y hemos vuelto escuchando un podcast en nuestros auriculares inalámbricos mientras andábamos, al llegar a casa y ponernos a preparar la comida nos gustaría poder seguir escuchando el podcast por el mismo minuto en el que íbamos pero en los altavoces de casa. Para hacer esto, actualmente podríamos conectar el móvil al altavoz usando el jack de 3.5, sin embargo los móviles actualmente carecen de dicho puerto (salvo pequeñas excepciones). Otra opción sería usar un asistente virtual (Alexa, Google Assistant, Siri) y pedirle que reproduzca el podcast que estábamos escuchando, el problema de esto es que tendremos que recordar el nombre concreto del podcast, tenemos que ponernos a hablarle al asistente lo cual rompe la sensación de integración y continuidad, y a todo esto el asistente tiene que ser capaz de entendernos correctamente, lo cual no pasa siempre. Otra opción más, la cual es compatible con la mayoría de altavoces con asistentes integrados, sería la de tener el altavoz en red y mandar el podcast desde el móvil a través de protocolos como Google Cast, AirPlay o Spotify Connect, lo cual es una de las opciones más rápidas y sencillas, pero implica tener que estar navegando a través de aplicaciones e interfaces hasta llegar a la opción de enviar contenido. Y en el caso de que tengamos muchos altavoces en red, tendremos que elegir de entre una lista amplia de dispositivos, a cual queremos enviar dicho contenido. De nuevo, este último caso, aún siendo el más factible, rompe mucho con la sensación de integración y continuidad de la que hablé anteriormente. Sin embargo, Apple si que tiene una solución para esto integrada en su ecosistema. Para los usuarios que tengan un iPhone y un HomePod, permiten que la reproducción del iPhone continúe en el HomePod simplemente con acercar el iPhone al HomePod, sin la necesidad de navegar por interfaces ni seleccionar nada. Todo de forma transparente para el usuario y en cuestión de segundos. El problema de esto es que fuera de ese ecosistema cerrado de Apple, no hay ninguna opción más. Si un usuario quiere disfrutar de esa transparencia y simplicidad, y no tiene iPhone, bien por el precio del mismo o bien porque prefiere tener un sistema más abierto como es Android, no le será posible disfrutar de esa opción que mantenga la sensación de integración y continuidad. A su vez, esta opción limita la posibilidad de escucha a únicamente el altavoz del HomePod o de los dispositivos HomePods que haya emparejados en casa. Por lo que en caso de tener un equipo propio de alta fidelidad o ajeno al ecosistema de Apple, tampoco se podrá usar.\\

Por tanto, la motivación de este proyecto es la de poder brindar dicha sensación de continuismo al sistema operativo Android con un coste notablemente más bajo que el de la solución de Apple. Y es que esa sensación de la que tanto hincapié se ha hecho a lo largo de la descripción, no es más que una forma de decir facilidad de uso, simpleza, accesibilidad y funcionamiento automático e independiente. Los cuales van a ser parte de los objetivos principales que guíen este proyecto (más adelante se hablará más en concreto de los objetivos). Por otro lado, además de poder solucionar dicho problema en el ecosistema de Android, con este proyecto también se va a buscar la posibilidad de usar altavoces que no sean HomePod.


\section{Motivación}
Por tanto, una vez puestos en contexto, la motivación de este proyecto es la de poder brindar dicha sensación de continuismo al sistema operativo Android con un coste notablemente más bajo que el de la solución de Apple. Y es que esa sensación de la que tanto hincapié se ha hecho a lo largo de la descripción, no es más que una forma de decir facilidad de uso, simpleza, accesibilidad y funcionamiento automático e independiente. Los cuales van a ser parte de los objetivos principales que guíen este proyecto (más adelante se hablará más en concreto de los objetivos). Por otro lado, además de poder solucionar dicho problema en el ecosistema de Android, con este proyecto también se va a buscar la posibilidad de usar altavoces que no sean HomePod. 

\section{Objetivos}
\begin{itemize}
    \item Voy a desarrollar dos reproductores de audio, uno para Android y otro para Linux, que sean capaces de comunicarse entre ellos. Ambos accederán a la misma biblioteca y soportarán audio en formato MP3.
    \item La comunicación entre ambos deberá de realizarse en menos de 3 segundos.
    \item Facilidad de uso y sencillez a la hora de compartir la música del reproductor al altavoz. (Se entiende que el proceso de montaje de la Raspberry Pi con su módulo NFC queda excluido de este objetivo, aún así se procurará facilitar el proceso de instalación lo más posible, ofreciendo opciones en Docker para desplegar todo, o incluso imagen .iso con el sistema operativo configurado lista para instalar en la SD e insertar en la Raspberry Pi)
    \item El coste final del proyecto debe estar por debajo de los 349€, el cual es el precio del HomePod a secas.
\end{itemize}

\section{Personajes}
Los personajes objetivo que tienen el problema presentado anteriormente y que por tanto son los que usarán la aplicación para solventarlo son:
\begin{itemize}
    \item María, 55 años, le gusta la música pero no se fía de introducir altavoces con asistentes virtuales en su hogar, tampoco se fía de las grandes empresas tecnológicas por el uso tan masivo que hacen de los datos y intromisión de las mismas en todo lo cotidiano. Fiel amante del Open Source.
    \item Paco, 75 años, desde pequeño siempre le ha gustado la ciencia ficción, y siempre ha admirado lo que la informática es capaz de hacer, le parece magia, sin embargo se lleva muy mal con ella, se le hace muy complicada y nunca es capaz de aprender correctamente a usarla.
    \item Victor, 18 años, le encanta la perfecta integración que tiene Apple dentro de su ecosistema, y siempre se traga todos los Apple Event, pero no tiene trabajo y no puede permitirse el precio tan elevado de tener el ecosistema Apple completo.
    \item El Piso de Olga es un grupo de jóvenes de entre 20 y 25 años que están acostumbrados a usar la tecnología pero quieren unificar el control y la interacción de la reproducción de música para todos los integrantes del grupo en un mismo dispositivo de forma simple y sin tener que prestarse los móviles.
\end{itemize}