\chapter{Introducción}

Este proyecto es software libre, y está liberado con la licencia \cite{gplv3}.\\

Puede encontrarse en el siguiente repositorio:
\url{https://github.com/migueorg/One-touch-music-streaming-TFG-ETSIIT/}, en el
cual se ha desarrollado desde el principio en abierto.

\section{Descripción del problema}
En los tiempos en los que estamos, cada vez es más normal escuchar música o
podcasts de fondo mientras hacemos otras tareas de nuestro día a día. Escuchamos
contenido mientras trabajamos, vamos en el coche, de camino al trabajo, de
vuelta a casa, mientras cocinamos, limpiamos, hay quienes también escuchan algún
tipo de podcast, ruido blanco o sonido para irse a dormir, entre muchos otros
más escenarios. 

A su vez, en el contexto informático en el que estamos, también le hemos dado la
espalda a los cables siempre que nos sea posible. Los auriculares inalámbricos
son un obligatorio en el día a día de muchas personas, facilitando así la
integración de la música en cada vez más contextos diferentes de nuestra vida.
Sin embargo, aún teniendo en cuenta los avances en el campo del \emph{IoT}, no
es tan fácil mantener esa sensación de integración y continuidad respecto a la
escucha de música. Ya que si queremos seguir escuchando en algún altavoz
inteligente la misma canción o podcast que veníamos escuchando en el móvil,
tendremos que empezar a interactuar con menús, dictar comandos (que no siempre
funcionan) al asistente virtual; o alguna otra alternativa compleja o cara.\\

\section{Motivación}
Por tanto, una vez puestos en contexto, la motivación de este proyecto es la de
poder brindar dicha sensación de continuidad al sistema operativo \emph{Android}
con un coste notablemente más bajo que el de la solución de \emph{Apple}. Y es
que esa sensación de la que tanto hincapié se ha hecho a lo largo de la
descripción, no es más que una manera de decir facilidad de utilización,
simpleza, accesibilidad y funcionamiento automático e independiente. Los cuales
van a ser parte de los objetivos principales que guíen este proyecto (más
adelante se hablará más en concreto de los objetivos). Por otro lado, además de
poder solucionar dicho problema en el ecosistema de \emph{Android}, con este
proyecto también se va a buscar la posibilidad de emplear altavoces que no sean
\emph{HomePod}. 

Esto se hará creando un servicio dotado por un emisor y un receptor de audio, el
cual sea capaz de enviar el sonido a dicho receptor. En el receptor se podrá
conectar altavoces de un tercero, ya sea por cable, jack, \emph{Bluetooth} o
cualquier vía que sea compatible con el dispositivo en donde se instale el
receptor.

\section{Objetivos}
\begin{itemize}
    \item Se deberá poder transmitir o mandar el sonido que se esté
    reproduciendo en el dispositivo móvil hasta el dispositivo que haga de
    altavoz o receptor.
    \item La comunicación entre ambos dispositivos (móvil y altavoz) deberá de
    llevarse a cabo en menos de 3 segundos para poder mantener la sensación de
    continuidad de la que hemos hablado antes. Más de 3 segundos puede hacer
    creer al usuario que el servicio no está funcionando.
    \item Facilidad de uso y sencillez a la hora de compartir el contenido del
    reproductor al altavoz. El proceso de envío de sonido desde el Android hasta
    el altavoz ha de poder realizarse sin interactuar con menús.
    \item El coste final del producto completo debe estar por debajo de los
    349€, el cual es el precio del HomePod a secas. (El producto completo
    estaría formado por móvil emisor + dispositivo receptor + accesorios de
    comunicación necesarios.)
\end{itemize}

\section{Personajes y viajes de usuario}
Los personajes que tienen el problema presentado anteriormente y que, por tanto,
son los que usarán la aplicación para solventarlo son:
\begin{itemize}
    \item Miguel, 52 años, es un contable al que le gusta la tecnología y tiene
    su casa llena de ordenadores viejos, equipos de música antiguos, altavoces,
    etc. Se ha ido adaptando a los cambios de la tecnología con los años, y como
    tal tiene su casa llena de altavoces inteligentes de los que se
    comercializan a día de hoy; aun así, echa de menos la calidad de los equipos
    de antes. Otro inconveniente que ve es que al tener tantos altavoces
    inteligentes conectados en red, cuando va a enviar el contenido al que tiene
    enfrente tiene que elegir entre una extensa lista en la cual, en ocasiones,
    le cuesta ubicarse; en especial cuando vuelve de trabajar escuchando un
    podcast, el cual desea terminarlo mientras prepara la comida. Sin embargo,
    ese proceso de continuar escuchando su podcast justo por donde iba en un
    altavoz inteligente cuando llega a casa es bastante tosco y molesto, pues le
    requiere tener que navegar entre los menús de compartir contenido y
    seleccionar el altavoz que tiene más cerca entre los tantos de su casa.
    Usando este proyecto Miguel desea poder continuar su escucha en sus antiguos
    (pero mejores) altavoces con solo acercar el móvil al dispositivo que haga
    de receptor.
\end{itemize}



\section{Historias de usuario}
A partir de los personajes y los user story presentados anteriormente, podemos
extraer ciertas funcionalidades del proyecto que el público destino espera
obtener o poder hacer. Estas funcionalidades clave tienen forma de las
siguientes historias de usuario:

\begin{enumerate}
    \item
    \href{https://github.com/migueorg/One-touch-music-streaming-TFG-ETSIIT/issues/10}{HU1}:
    Como usuario que busca la mayor automatización y transparencia con la
    tecnología quiero poder seguir escuchando mi reproducción actual manteniendo
    el mismo momento de escucha sin tener que interactuar con ningún menú al
    enviar hacia mis propios altavoces y equipos viejos, con una inversión baja.
    \item
    \href{https://github.com/migueorg/One-touch-music-streaming-TFG-ETSIIT/issues/14}{HU4}:
    Como persona ajena al proyecto quiero conocer las decisiones que se han
    tomado al desarrollar el proyecto para poder entender por qué se han
    empleado ciertas plataformas y componentes y poder continuar o modificar su
    desarrollo si fuese necesario.
\end{enumerate}

Nótese que además de las funcionalidades que el usuario final espera obtener,
también se ha dedicado una historia de usuario (HU5) que personifica al tribunal o
a cualquier persona que en el futuro encuentre este proyecto y quiera conocer
por qué se ha actuado de la forma que se ha hecho durante todo el trabajo.

\section{Milestones}
Los productos mínimamente viables o \emph{releases} que se irán haciendo durante el
desarrollo se han estimado que serán los siguientes:

\begin{enumerate}
    \item \href{https://github.com/migueorg/One-touch-music-streaming-TFG-ETSIIT/milestone/1}{M0}: Repositorio y documentación funcional sobre la que empezar a programar.
    \item \href{https://github.com/migueorg/One-touch-music-streaming-TFG-ETSIIT/milestone/2}{M1}: Captar el audio del teléfono Android para transmitirlo.
    \item \href{https://github.com/migueorg/One-touch-music-streaming-TFG-ETSIIT/milestone/3}{M2}: Recibir en un dispositivo diferente el audio previamente transmitido.
    \item \href{https://github.com/migueorg/One-touch-music-streaming-TFG-ETSIIT/milestone/4}{M3}: Automatizar la tarea de envío y recepción en un solo gesto/toque
    \item \href{https://github.com/migueorg/One-touch-music-streaming-TFG-ETSIIT/milestone/5}{M4}: Memoria entregable para el tribunal.
    \item \href{https://github.com/migueorg/One-touch-music-streaming-TFG-ETSIIT/milestone/6}{M5}: Presentación.
\end{enumerate}