\chapter{Introducción}

Este proyecto es software libre, y está liberado con la licencia \cite{gplv3}.\\

Puede encontrarse en el siguiente repositorio:
\url{https://github.com/migueorg/One-touch-music-streaming-TFG-ETSIIT/}, en el
cual se ha desarrollado desde el principio en abierto.

\section{Descripción del Problema}
En los tiempos en los que estamos, cada vez es más normal escuchar música o
podcasts de fondo mientras hacemos otras tareas de nuestro día a día. Escuchamos
contenido mientras trabajamos, vamos en el coche, de camino al trabajo, de
vuelta a casa, mientras cocinamos, limpiamos, hay quienes también escuchan algún
tipo de podcast, ruido blanco o sonido para irse a dormir, entre muchos otros
más escenarios. 

A su vez, en el contexto informático en el que estamos,
también le hemos dado la espalda a los cables siempre que nos sea posible. Los
auriculares inalámbricos son un obligatorio en el día a día de muchas personas,
facilitando así la integración de la música en cada vez más contextos diferentes
de nuestra vida. Sin embargo, aún teniendo en cuenta los avances en el campo del
\emph{IoT}, no es tan fácil mantener esa sensación de integración y continuidad
respecto a la escucha de música.\\

Por ejemplo si llegamos a casa del trabajo, y hemos vuelto escuchando un podcast
en nuestros auriculares inalámbricos mientras andábamos, al llegar a casa y
ponernos a preparar la comida nos gustaría poder seguir escuchando el podcast
por el mismo minuto en el que íbamos pero en los altavoces de casa. 

Para hacer esto, actualmente podríamos conectar el móvil al altavoz usando el
\emph{jack} de 3.5, sin embargo, los móviles actualmente carecen de dicho puerto
(salvo pocas excepciones). 

Otra opción sería emplear un asistente virtual (\emph{Alexa, Google Assistant,
Siri}) y pedirle que reproduzca el podcast que estábamos escuchando, el problema
de esto es que tendremos que recordar el nombre concreto del podcast, tenemos
que ponernos a hablarle al asistente lo cual rompe la sensación de integración y
continuidad, y a todo esto el asistente tiene que ser capaz de entendernos
correctamente, lo cual no pasa siempre. 

Otra opción más, la cual es compatible con la mayoría de altavoces con
asistentes integrados, sería la de tener el altavoz en red y mandar el podcast
desde el móvil a través de protocolos como \emph{Google Cast, AirPlay o Spotify
Connect}, lo cual es una de las opciones más rápidas y sencillas, pero implica
tener que estar navegando a través de aplicaciones e interfaces hasta llegar a
la opción de enviar contenido. Y en el caso de que tengamos muchos altavoces en
red, tendremos que elegir de entre una lista amplia de dispositivos, al cual
queremos enviar dicho contenido. De nuevo, este último caso, aún siendo el más
factible, rompe mucho con la sensación de integración y continuidad de la que
hablé anteriormente. 

Sin embargo, \emph{Apple} sí que tiene una solución para esto integrada en su
ecosistema. Para los usuarios que tengan un \emph{iPhone} y un \emph{HomePod},
permiten que la reproducción del \emph{iPhone} continúe en el \emph{HomePod}
simplemente con acercar el \emph{iPhone} al \emph{HomePod}, sin la necesidad de
navegar por interfaces ni seleccionar nada. Todo de forma transparente para el
usuario y en cuestión de segundos. El problema de esto es que fuera de ese
ecosistema cerrado de \emph{Apple}, no hay ninguna opción más. Si un usuario
quiere disfrutar de esa transparencia y simplicidad, y no tiene \emph{iPhone},
bien por el precio del mismo o bien porque prefiere tener un sistema más abierto
como es \emph{Android}, no le será posible disfrutar de esa opción que mantenga
la sensación de integración y continuidad. A su vez, esta opción limita la
posibilidad de escucha a únicamente el altavoz del \emph{HomePod} o de los
dispositivos \emph{HomePods} que haya emparejados en casa. Por lo que en caso de
tener un equipo propio de alta fidelidad o ajeno al ecosistema de \emph{Apple},
tampoco se podrá usar.\\

\section{Motivación}
Por tanto, una vez puestos en contexto, la motivación de este proyecto es la de
poder brindar dicha sensación de continuidad al sistema operativo \emph{Android}
con un coste notablemente más bajo que el de la solución de \emph{Apple}. Y es
que esa sensación de la que tanto hincapié se ha hecho a lo largo de la
descripción, no es más que una manera de decir facilidad de utilización,
simpleza, accesibilidad y funcionamiento automático e independiente. Los cuales
van a ser parte de los objetivos principales que guíen este proyecto (más
adelante se hablará más en concreto de los objetivos). Por otro lado, además de
poder solucionar dicho problema en el ecosistema de \emph{Android}, con este
proyecto también se va a buscar la posibilidad de emplear altavoces que no sean
\emph{HomePod}. 

\section{Objetivos}
\begin{itemize}
    \item Se deberá poder transmitir o mandar el sonido que se esté
    reproduciendo en el dispositivo Android hasta el dispositivo que haga de
    altavoz.
    \item La comunicación entre ambos dispositivos (Android y altavoz) deberá de
    llevarse a cabo en menos de 3 segundos para poder mantener la sensación de
    continuidad de la que hemos hablado antes. Más de 3 segundos puede hacer
    creer al usuario que el servicio no está funcionando.
    \item Facilidad de uso y sencillez a la hora de compartir el contenido del
    reproductor al altavoz. El proceso de envío de sonido desde el Android hasta
    el altavoz ha de poder realizarse sin interactuar con menús. (Se entiende
    que el proceso de montaje de la Raspberry Pi con su módulo NFC queda
    excluido de este objetivo, aun así se procurará facilitar el proceso de
    instalación lo más posible, ofreciendo opciones en Docker para desplegar
    todo, o incluso una imagen en formato ISO con el sistema operativo
    configurado lista para instalar en la tarjeta SD e insertar en la Raspberry
    Pi)
    \item El coste final del producto debe estar por debajo de los 349€, el cual
    es el precio del HomePod a secas.
\end{itemize}

\section{Personajes}
Los personajes que tienen el problema presentado anteriormente y que, por tanto,
son los que usarán la aplicación para solventarlo son:
\begin{itemize}
    \item María, 55 años, le gusta la música, pero no se fía de introducir
    altavoces con asistentes virtuales en su hogar; tampoco se fía de las
    grandes empresas tecnológicas por el uso tan masivo que hacen de los datos e
    intromisión de las mismas en todo lo cotidiano. Fiel amante del Open Source.
    \item Paco, 80 años, desde pequeño siempre le ha gustado la ciencia ficción,
    y siempre ha admirado lo que la informática es capaz de hacer, le parece
    magia, sin embargo, se lleva muy mal con ella, se le hace muy complicada y
    nunca es capaz de aprender correctamente a utilizarla.
    \item Víctor, 18 años, le encanta la perfecta integración que tiene Apple
    dentro de su ecosistema, y siempre se traga todos los Apple Event, no
    obstante, no tiene trabajo y no puede permitirse el precio tan elevado de
    tener el ecosistema Apple completo.
    \item El Piso de Olga es un grupo de jóvenes de entre 20 y 25 años que están
    acostumbrados a usar la tecnología, pero quieren unificar el control y la
    interacción de la reproducción de música para todos los integrantes del
    grupo en un mismo dispositivo de forma simple y sin tener que prestarse los
    móviles.
\end{itemize}

\section{User Journeys}
Los contextos en los que los personajes citados arriba usarán la aplicación son
los siguientes:\\

Persona que no se fía de los asistentes virtuales, sospecha que siempre le están
escuchando y espiando, así que no quiere altavoces inteligentes que tengan
micrófono, sin embargo, quiere poder seguir escuchando en un altavoz la música
que escucha en el móvil cuando llega a casa, sin tener que andar trasteando los
menús de las aplicaciones de música, ni tener que navegar por menús y pantallas
de dispositivos con altavoces en casa.\\

Persona mayor a la que le gusta la domótica y es fiel amante de la música, le
recuerda a eso que siempre vio en las películas cuando era pequeño, sin embargo,
cada vez que ha intentado adentrarse en ella (la domótica) le abruma lo
complicada que es, nunca se entiende con los asistentes virtuales, nunca sabe
que tiene que decir para que le entiendan, incluso no le entienden bien, ya que,
tiene un acento muy cerrado así que preferiría algo más sencillo que los
asistentes virtuales. Le gustaría poder escuchar música de su biblioteca en el
móvil o en sus altavoces de forma sencilla y sin cables, discos, ni historias.
Su hijo informático ya ha probado a ponerle en casa diferentes equipos
inteligentes, pero ninguno se acomoda a él.\\

Joven que le gusta el ecosistema de Apple y ve las comodidades que este ofrece,
como poder transferir contenido desde el iPhone al HomePod, sin embargo, al no
tener trabajo no se puede permitir los precios de un HomePod, un iPhone y una
suscripción mensual a Apple Music.\\

Piso Olga quieren que cuando organicen fiestas, sus invitados puedan cada uno
añadir canciones a la cola fácilmente con sus Android, sin tener que estar
dándole continuamente el móvil del que tenga conectado el altavoz a todos cuando
están bebidos.\\

\section{Historias de Usuario}
A partir de los personajes y los user story presentados anteriormente, podemos
extraer ciertas funcionalidades del proyecto que el público destino espera
obtener o poder hacer. Estas funcionalidades clave tienen forma de las
siguientes historias de usuario:

\begin{enumerate}
    \item HU0: Como programador quiero saber detalladamente que problema tengo
    que solventar para empezar a programar.
    \item HU1: Como usuario con pocos recursos económicos (Víctor) quiero poder
    ejecutar el reproductor en un teléfono Android para poder usar la aplicación
    en un móvil barato.
    \item HU2: Como usuario amante de la música y con equipo de audio propio
    (María) quiero utilizar mi propio equipo de altavoces sin tener que utilizar
    un altavoz comercial de alguna otra marca para escuchar el contenido que
    envío desde el móvil.
    \item HU3: Como usuario con experiencias nefastas en otros sistemas
    domóticos, quiero enviar el contenido de la manera más sencilla posible para
    no tener que andar pulsando botones ni buscando menús de compartir.
    \item HU4: Como grupo de usuarios que empleará la aplicación con diferentes
    móviles en un solo altavoz (Piso Olga) queremos poder acceder todos al mismo
    catálogo de audio para no tener que andar pasando ficheros al altavoz.
    \item HU5: Como persona ajena al proyecto quiero conocer las decisiones que
    se han tomado al desarrollar el proyecto para poder entender por qué se han
    empleado ciertas plataformas y componentes y poder continuar o modificar su
    desarrollo si fuese necesario.
\end{enumerate}

Nótese que además de las funcionalidades que el usuario final espera obtener,
también se ha dedicado una Historia de Usuario (HU0) para el papel del
programador, en este caso yo, con lo que esperaría tener a la hora de empezar un
proyecto. Así como otra Historia de Usuario (HU5) que personifica al tribunal o
a cualquier persona que en el futuro encuentre este proyecto y quiera conocer
por qué se ha actuado de la forma que se ha hecho durante todo el trabajo.

\section{Milestones}
Los productos mínimamente viables o releases que se irán haciendo durante el
desarrollo se han estimado que serán los siguientes:

\begin{enumerate}
    \item M0: Repositorio y documentación funcional sobre la que empezar a programar.
    \item M1: Reproductor de música local para Android.
    \item M2: Reproductor de música para el altavoz (Raspberry Pi) basado en Linux.
    \item M3: Comunicación NFC entre Android y Raspberry Pi.
    \item M4: Biblioteca de música remota para el reproductor de Android.
    \item M5: Documentación.
    \item M6: Presentación.
\end{enumerate}