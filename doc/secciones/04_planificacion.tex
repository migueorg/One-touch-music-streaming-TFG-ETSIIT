\chapter{Planificación}

La planificación es algo esencial para todos los proyectos, y juega un papel aún
más importante si cabe, cuando se involucra el desarrollo ágil, como es el caso.

Por tanto, en este capítulo detallaremos que metodolgías se van a seguir, como
se va a organizar temporalmente el proyecto, y como se ha ido siguiendo el
desarrollo y asegurando la calidad del mismo.

\section{Metodología utilizada}

A continuación documentaremos que herramientas y métodos se ha ido empleando
durante el proyecto para ir guiando su desarrollo. Asegurando de esta forma, que
con el tiempo no se pierda el enfoque y los objetivos iniciales.

\subsection{Corrector ortográfico}
Para poder asegurar la calidad, no solo del código, sino también del documento
entregable en cuanto a calidad gramatical y ortográfica se refiere, se ha
implementado un corrector que verificará en todo momento que cualquier adición a
la documentación está libre de faltas. 

Para elegir dicho corrector entre todas las posibles alternativas que existen de
la misma, se ha seguido un proceso de selección en el cual se establecerán
inicialmente unos criterios de búsqueda con el que encontrar las herramientas.
Posteriormente se evaluarán todas las herramientas encontradas usando los
criterios de selección previamente establecidos.:

\subsubsection{Criterios de búsqueda}

Los criterios a través de los cuales buscaremos los candidatos para
posteriormente evaluar si cumplen los requisitos de selección:
\begin{itemize}
    \item Deben de incluir el idioma español. No todos los \emph{spell checkers}
    incluyen el idioma español, y si lo incluyen suelen ser de manera muy
    limitada y no tan completa como el idioma base en el que fueron programados.
    \item Deben poderse añadir como \emph{GitHub Action} al repositorio para así
    integrarlos en forma de test, y solamente pase si el texto está correcto.
    \item Debe ser compatible con LaTeX.
\end{itemize}

\subsubsection{Criterios de selección}

Los criterios bajo los cuales evaluaremos los candidatos encontrados en la
búsqueda anterior para finalmente seleccionar uno son:
\begin{itemize}
    \item Debe tener una comunidad activa por si es necesario hacer alguna
    pregunta o consulta en el futuro.
    \item Los informes sobre los fallos existentes deben ser lo más legible y
    fácil de entender posible.
    \item Se valorará que ofrezca posibles correcciones.
    \item Se valorará la posibilidad de excluir palabras del análisis. 

\end{itemize}


\subsubsection{Opciones encontradas según los criterios de búsqueda establecidos}

Tras buscar \emph{spell checkers} que cumplan los requisitos de búsqueda son:

\begin{itemize}
    \item \href{https://github.com/marketplace/actions/textidote-action}{TeXtidote Action}
\end{itemize}

Desafortunadamente, únicamente he podido encontrar un \emph{spell checker} que
cumpliese todos los requisitos de búsqueda previamente establecidos. Ya que otros
encontrados \textbf{no cumplían algunos de los requisitos}:
\begin{itemize}
    \item \href{https://github.com/marketplace/actions/cspell-action}{cspell-action} Pensado principalmente para código no para documentos LaTeX. Nada de español.
    \item \href{https://github.com/marketplace/actions/spell-checker-action}{Spell Checker Action}: Compatible únicamente con inglés. No está centrado en LaTeX. 
    \item \href{https://github.com/valentjn/vscode-ltex}{VSCode LTeX}: No dispone de GitHub Action.
    \item \href{https://github.com/erodner/latex-checks}{LaTeX-Checks}: No dispone de GitHub Action. Solo para inglés y alemán.
\end{itemize}

\subsubsection{Proceso de selección y estudio}
Dado que solo hay un corrector ortográfico a analizar, el proceso de estudio y
decisión entre las alternativas no es estrictamente necesario, pero igualmente
se realizará según los criterios de selección previamente establecidos para ver
en que lugar queda: 

\begin{todolist}
    \item[\xcmark] Debe tener una comunidad activa por si es necesario hacer
    alguna pregunta o consulta en el futuro. Cumple este requisito pues se
    mantiene activo en GitHub resolviendo los issues y PR de la comunidad.
    \item[\xcmark] Los informes sobre los fallos existentes deben ser lo más
    legible y fácil de entender posible. Cumple este requisito pues genera un
    reporte en HTML con los problemas resaltados en colores llamativos.
    \item[\xcmark] Se valorará que ofrezca posibles correcciones. Cumple este
    requisito pues en dicho reporte al poner el cursor encima te sugiere
    revisiones.
    \item[\xcmark] Se valorará la posibilidad de excluir palabras del análisis.
    Cumple el requisito aunque bastante raspado, ya que permite excluir palabras
    mediante un diccionario de palabras excluidas, pero no permite excluir una
    única aparición puntual de la misma.
\end{todolist}

\subsubsection{Conclusión}
Finalmente, dado que es la única alternativa que se ajusta completamente a los
requisitos de búsqueda y además cumple en mayor o menor medida con los
requisitos de selección establecidos, se ha implementado esta solución como
GitHub Action en el repositorio, para asegurar la calidad del documento
entregable.

\section{Temporización}

\section{Seguimiento del desarrollo}
