\chapter{Planificación}

\section{Metodología utilizada}

\subsection{Corrector Ortográfico}
Para poder asegurar la calidad, no solo del código, sino también del documento
entregable en cuanto a calidad gramatical y ortográfica se refiere, se ha
implementado un corrector que verificará en todo momento que cualquier adición a
la documentación está libre de faltas. 

Para elegir dicho corrector entre todas las posibles alternativass que existen
de la misma, se ha seguido un proceso de selección:

\subsubsection{Criterios de búsqueda}

Los criterios a través de los cuales buscaremos los candidatos para
posteriormente evalar si cumplen los requisitos de selección:
\begin{itemize}
    \item Deben de incluir el idioma español. No todos los \emph{spell checkers}
    incluyen el idioma español, y si lo incluyen suelen ser de forma muy
    limitada y no tan completa como el idioma base en el que fuero programados.
    \item Deben poderse añadir como \emph{GitHub Action} al repositorio para de esta
    manera integrarlos en forma de test, y solo pase si el texto está correcto.
    \item Debe ser compatible con LaTeX.
\end{itemize}

\subsubsection{Criterios de selección}

Los criterios bajo los cuales evalaremos los candidatos encontrados en la
búsqueda anterior para finalmente seleccionar uno son:
\begin{itemize}
    \item Debe tener una comunidad activa por si es necesario hacer alguna
    pregunta o consulta en el futuro.
    \item Los informes sobre los fallos existentes debe ser lo más legibles y
    fácil de entender posible.
    \item Se valorará que ofrezca posibles correcciones.
    \item Se valorará la posibilidad de excluir palabras del analisis. 

\end{itemize}


\subsubsection{Opciones encontradas según los criterios de búsqueda establecidos}

Tras buscar \emph{Spell Checkers} que cumplan los requisitos de búsqueda son:

\begin{itemize}
    \item 
\end{itemize}


\section{Temporización}

\section{Seguimiento del desarrollo}
