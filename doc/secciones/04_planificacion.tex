\chapter{Planificación}

La planificación es algo esencial para todos los proyectos, y juega un papel aún 
más importante si cabe, cuando se involucra el desarrollo ágil, como es el caso. 

Por tanto, en este capítulo detallaremos que metodologías se van a seguir, como 
se va a organizar temporalmente el proyecto, y como se ha ido siguiendo el 
desarrollo y asegurando la calidad del mismo. 

\section{Metodología utilizada}

A continuación documentaremos que herramientas y métodos se ha ido empleando 
durante el proyecto para ir guiando su desarrollo. Asegurando de esta forma, que 
con el tiempo no se pierda el enfoque y los objetivos iniciales. 

\subsection{Corrector ortográfico}
Para poder asegurar la calidad que buscamos siguiendo los principios del manifiesto
ágil \cite{agile-manifiesto}, no solo del código, sino también del documento
entregable en cuanto a calidad gramatical y ortográfica se refiere, se ha
implementado un corrector que verificará en todo momento que cualquier adición a
la documentación está libre de faltas. 

Para elegir dicho corrector entre todas las posibles alternativas que existen de 
la misma, se ha seguido un proceso de selección en el cual se establecerán 
inicialmente unos criterios de búsqueda con el que encontrar las herramientas. 
Posteriormente, se evaluarán todas las herramientas encontradas usando los 
criterios de selección previamente establecidos.: 

\subsubsection{Criterios de búsqueda}

Los criterios a través de los cuales buscaremos los candidatos para
posteriormente evaluar si cumplen los requisitos de selección:
\begin{itemize}
    \item Deben de incluir el idioma español. No todos los \emph{spell checkers}
    incluyen el idioma español, y si lo incluyen suelen ser de manera muy
    limitada y no tan completa como el idioma base en el que fueron programados.
    \item Deben poderse añadir como \emph{GitHub Action} al repositorio para así
    integrarlos en forma de test, y solamente pase si el texto está correcto.
    \item Debe ser compatible con LaTeX.
\end{itemize}

\subsubsection{Criterios de selección}

Los criterios bajo los cuales evaluaremos los candidatos encontrados en la
búsqueda anterior para finalmente seleccionar uno son:
\begin{itemize}
    \item Debe tener una comunidad activa por si es necesario hacer alguna
    pregunta o consulta en el futuro.
    \item Los informes sobre los fallos existentes deben ser lo más legible y
    fácil de entender posible.
    \item Se valorará que ofrezca posibles correcciones.
    \item Se valorará la posibilidad de excluir palabras del análisis. 

\end{itemize}


\subsubsection{Opciones encontradas según los criterios de búsqueda establecidos}

Tras buscar \emph{spell checkers} que cumplan los requisitos de búsqueda son:

\begin{itemize}
    \item \href{https://github.com/marketplace/actions/textidote-action}{TeXtidote Action}
\end{itemize}

Desafortunadamente, únicamente he podido encontrar un \emph{spell checker} que
cumpliese todos los requisitos de búsqueda previamente establecidos. Ya que otros
encontrados \textbf{no cumplían algunos de los requisitos}:
\begin{itemize}
    \item \href{https://github.com/marketplace/actions/cspell-action}{cspell-action} Pensado principalmente para código no para documentos LaTeX. Nada de español.
    \item \href{https://github.com/marketplace/actions/spell-checker-action}{Spell Checker Action}: Compatible únicamente con inglés. No está centrado en LaTeX. 
    \item \href{https://github.com/valentjn/vscode-ltex}{VSCode LTeX}: No dispone de GitHub Action.
    \item \href{https://github.com/erodner/latex-checks}{LaTeX-Checks}: No dispone de GitHub Action. Solo para inglés y alemán.
\end{itemize}

\subsubsection{Proceso de selección y estudio}
Dado que solo hay un corrector ortográfico a analizar, el proceso de estudio y
decisión entre las alternativas no es estrictamente necesario, pero igualmente
se realizará según los criterios de selección previamente establecidos para ver
en que lugar queda: 

\begin{todolist}
    \item[\xcmark] Debe tener una comunidad activa por si es necesario hacer
    alguna pregunta o consulta en el futuro. Cumple este requisito pues se
    mantiene activo en GitHub resolviendo los \emph{issues} y \emph{PR} de la
    comunidad.
    \item[\xcmark] Los informes sobre los fallos existentes deben ser lo más
    legible y fácil de entender posible. Cumple este requisito pues genera un
    reporte en \emph{HTML} con los problemas resaltados en colores llamativos.
    \item[\xcmark] Se valorará que ofrezca posibles correcciones. Cumple este
    requisito pues en dicho reporte al poner el cursor encima te sugiere
    revisiones.
    \item[\xcmark] Se valorará la posibilidad de excluir palabras del análisis.
    Cumple el requisito aunque bastante raspado, ya que permite excluir palabras
    mediante un diccionario de palabras excluidas, sin embargo, no permite
    excluir una única aparición puntual de la misma.
\end{todolist}

\subsubsection{Conclusión}
Finalmente, dado que es la única alternativa que se ajusta completamente a los
requisitos de búsqueda y además cumple en mayor o menor medida con los
requisitos de selección establecidos, se ha implementado esta solución como
\emph{GitHub Action} en el repositorio, para asegurar la calidad del documento
entregable.

\subsection{Test runner: Java y Kotlin}

\subsubsection{Criterios de búsqueda}

Los criterios a través de los cuales buscaremos los candidatos para
posteriormente evaluar si cumplen los requisitos de selección:
\begin{itemize}
    \item Se buscarán test runners que sean compatibles con Java y Kotlin.
    Puesto que vamos a trabajar con dos lenguajes distintos, pero que a su vez
    comparten muchas similitudes y bibliotecas, con el fin de ahorrar en tener
    que invertir tiempo en aprender y comprender dos frameworks de testing
    diferentes, buscaremos uno que se pueda incluir en ambos lenguajes,
    ahorrando de esta manera tiempo y código. 
    \item Se buscará que no estén centrados en BDD, ya que, algunos test runners
    están centrados en dicho enfoque y basan su sintaxis en el \emph{Scenario,
    Given, When y Then}, lo cual se aleja del enfoque DDD empleado en este proyecto.
\end{itemize}

\subsubsection{Criterios de selección}

Los criterios bajo los cuales evaluaremos los candidatos encontrados en la
búsqueda anterior para finalmente seleccionar el que más se adecúe son:
\begin{itemize}
    \item Se valorará el nivel de documentación que tenga así como de los
    ejemplos y tutoriales hechos por la comunidad.
    \item Se tendrá en cuenta la compatibilidad con IDE.
\end{itemize}


\subsubsection{Opciones encontradas según los criterios de búsqueda establecidos}

Tras buscar \emph{test runners} que cumplan los requisitos de búsqueda encontramos:

\begin{itemize}
    \item \href{https://junit.org/junit5/}{JUnit}
    \item \href{https://testng.org/doc/index.html}{TestNG}
\end{itemize}

Otros \emph{test runners} encontrados, pero que no cumplían los requisitos de búsqueda son:
\begin{itemize}
    \item \href{https://spockframework.org/}{Spock}: Centrado en BDD y \href{https://kotlinlang.org/api/latest/kotlin.test/}{no es soportado nativamente por Kotlin}, aunque se puede añadir gracias a Groovy.
    \item \href{https://cucumber.io/}{Cucumber} Centrado en BDD y \href{https://kotlinlang.org/api/latest/kotlin.test/}{no es soportado nativamente por Kotlin}.
    \item \href{https://github.com/karatelabs/karate}{Karate} Centrado en BDD y \href{https://kotlinlang.org/api/latest/kotlin.test/}{no es soportado nativamente por Kotlin}.
\end{itemize}


\subsubsection{Proceso de selección y estudio}

\subsubsection{JUnit}

\begin{todolist}
    \item[\xcmark] Se valorará el nivel de documentación que tenga así como de
    los ejemplos y tutoriales hechos por la comunidad: Cumple encarecidamente
    este requisito, ya que, es el \emph{test runner} por excelencia en Java, por
    tanto, es el más usado y el que más ejemplos, tutoriales y documentación
    posee.
    \item[\xcmark] Se tendrá en cuenta la compatibilidad con IDE: Cumple este
    requisito sin problemas, puesto que, existe
    \href{https://code.visualstudio.com/docs/java/java-testing}{esta extensión para
    Visual Studio Code} que lo hace compatible y además
    \href{https://www.jetbrains.com/help/idea/junit.html}{viene por defecto con
    IntelliJ IDEA}
\end{todolist}


\subsubsection{TestNG}

\begin{todolist}
    \item Se valorará el nivel de documentación que tenga así como de los
    ejemplos y tutoriales hechos por la comunidad: A pesar de que hay bastante
    bibliografía y ejemplos, no está al nivel de JUnit y de hecho incluso la
    página oficial tiene un aspecto más antiguo y descuidado que el de JUnit.
    \item[\xcmark] Se tendrá en cuenta la compatibilidad con IDE: Cumple este
    requisito sin problemas, pues, existe
    \href{https://code.visualstudio.com/docs/java/java-testing}{esta extensión para
    Visual Studio Code} que lo hace compatible y además
    \href{https://www.jetbrains.com/help/idea/testng.html}{viene por defecto con
    IntelliJ IDEA}
\end{todolist}


Por lo que tras valorar mediante los criterios de selección los \emph{test
runners} principales encontrados según los criterios de búsqueda, podemos ver
que el que satisface todos los criterios es \textbf{JUnit}. Por lo que este será
el \emph{test runner} empleado para Java y para Kotlin.


\subsection{Biblioteca de aserciones: Java y Kotlin}

La elección de una biblioteca de aserciones no es excluyente de prescindir de
otras, ya que, puede haber conjuntamente más de una a la vez en un mismo
proyecto. Por lo que se empleará la resulte más conveniente según lo que se esté
testeando. Así pues este proceso de selección no buscará quedarse con una sola
opción, sino consultar las existentes para tenerlas en conocimiento y poder
optar por una o por otra en el momento en el que se necesiten.

\subsubsection{Criterios de búsqueda}

Los criterios a través de los cuales buscaremos los candidatos para
posteriormente evaluar si cumplen los requisitos de selección:
\begin{itemize}
    \item Bibliotecas de aserciones que sean compatibles con JUnit (el
    \emph{test runner} que estamos usando)
\end{itemize}

\subsubsection{Resultados tras la búsqueda}

Las diferentes opciones que se han encontrado compatibles con nuestros criterios
de búsqueda son:

\begin{itemize}
    \item La biblioteca que proporciona JUnit: Básica y con una sintaxis poco
    natural. Pero suficiente para cosas simples o sencillas.
    \item \href{https://github.com/assertj/assertj}{AssertJ}: Mucho más completa
    que la que incorpora JUnit, con sintaxis mucho más natural y con la
    posibilidad de crear tus propias aserciones. Es una de las más empleadas.
    \item \href{http://www.awaitility.org/}{Awaitility}: Se centra en brindar y
    facilitar los test de manera legible para sistemas asíncronos.
    \item \href{https://site.mockito.org/}{Mockito}: Utilizado por excelencia para
    ``mockear'', tiene la ventaja de que se adapta tanto a escenarios simples
    como para escenarios más complejos y avanzados o específicos. Se suele
    añadir en conjunto a AssertJ y es otro de los más frecuentados por la comunidad.
    \item \href{https://github.com/voodoodyne/subethasmtp}{Wiser}: Librería de
    uso exclusivo para ``mockear'' un servidor SMTP. 
    \item \href{https://github.com/marschall/memoryfilesystem}{Memoryfilesystem}
    y \href{https://github.com/google/jimfs}{Jimfs}: Librerías para
    ``mockear'' entrada y salida de memoria. 
    \item \href{https://wiremock.org/docs/overview/}{WireMock}: Librería para
    ``mockear'' servidores HTTP
    \item \href{https://testcontainers.com/}{Testcontainers}: Framework muy
    interesante para ``mockear'' cualquier cosa que pueda ejecutarse en un
    contendor Docker. 
\end{itemize}

Se han añadido librerías para ``mockear'' a pesar de no ser bibliotecas de
aserciones como tal, pero que pueden ser requeridas igualmente según lo que se
esté testeando. Igualmente, existirán más librerías y frameworks específicos para hacer
\emph{mocks} más concretos según necesidad.



\subsection{Test runner: Python}

\subsubsection{Criterios de búsqueda}

Los criterios a través de los cuales buscaremos los candidatos para
posteriormente evaluar si cumplen los requisitos de selección:
\begin{itemize}
    \item Se buscará cualquier test runner que sea compatible con Python.
\end{itemize}

\subsubsection{Criterios de selección}

Los criterios bajo los cuales evaluaremos los candidatos encontrados en la
búsqueda anterior para finalmente seleccionar el que más se adecúe son:
\begin{itemize}
    \item Comunidad Activa (se tendrá en cuenta la frecuencia con la que hacen
    commits, fecha de la última versión, cantidad de \emph{PR} e \emph{issues}
    abiertos, forma de contestar ante aportaciones de la comunidad, número de
    personas que mantienen el proyecto, etc.)
    \item Claridad en los informes de los test (se valorará la salida que nos dé
    al finalizar los test, la facilidad para ver que ha salido bien y que ha
    salido mal, etc.)
    \item Documentación concisa (se valorarán los ejemplos proporcionados en la
    documentación, organización de las secciones, la facilidad para buscar dudas
    o problemas, existencia de FAQ, así como utilidad de la misma)
    \item Facilidad de uso (se tendrá en cuenta la curva inicial de dificultad
    para usar el framework, simplicidad para desplegarlo así como la complejidad
    para añadir los test)
    \item Debe tener la posibilidad de usar fixtures, pues aunque no se vayan a
    usar de momento, es un añadido que en el futuro puede ser útil para poder
    simular contextos o situaciones específicas para testear comportamientos
    concretos de forma cómoda y rápida.
\end{itemize}


\subsubsection{Opciones encontradas según los criterios de búsqueda establecidos}

Tras buscar \emph{test runners} que cumplan los requisitos de búsqueda encontramos:

\begin{itemize}
    \item \href{https://github.com/pytest-dev/pytest}{Pytest}
    \item \href{https://github.com/nose-devs/nose2}{Nose2}
    \item \href{https://github.com/CleanCut/green}{Green}
    \item \href{https://github.com/nestorsalceda/mamba}{Mamba}
\end{itemize}

\subsubsection{Proceso de selección y estudio}

\subsubsection{Pytest}

\begin{todolist}
    \item[\xcmark] Cumple con una comunidad activa. De hecho al ser uno de los
    más conocidos, tiene una gran actividad tanto en \emph{issues} como en
    \emph{PR}, y en medida de lo posible por el gran volumen, se ve como
    contestan con rapidez y los atienden. Además, también tienen una gran
    cantidad de commits, todos con bastante frecuencia, y el equipo de
    desarrollo no está compuesto o mantenido por una sola persona.
    \item[\xcmark] Dispone de una salida bastante fácil de comprender y rápida
    de detectar el error, resaltando la línea que provoca el fallo y el motivo
    de una manera ordenada e intuitiva, por lo que sí que cumple este requisito.
    \item A pesar de tener una documentación correcta, no tiene ejemplos
    concretos ni un modo organizado de mostrarla, así como explicaciones
    suficientes que acompañen a algunos de los ejemplos. Por tanto, no cumple
    este criterio.
    \item[\xcmark] Los pasos iniciales son bastante simples y rápidos, como
    consecuencia, tiene una curva inicial bastante simple a pesar de una
    documentación que deja que desear, por lo que si cumple este requisito.
    \item[\xcmark]
    \href{https://docs.pytest.org/en/latest/explanation/fixtures.html}{Soporta
    el uso de fixtures}, por lo que si cumple este requisito. 
\end{todolist}


\subsubsection{Nose2}

\begin{todolist}
    \item No cumple con mis criterios de tener una comunidad activa, a pesar de
    que su commit más reciente es de hace 2 días y no tienen muchos
    \emph{issues} ni \emph{PR} abiertos, pero realmente sus commits son de muy
    cuando en cuando, por lo que se aprecia que el desarrollo va por rachas y no
    es demasiado activo, y además de esto, solo lo mantiene activo una persona.
    Por lo que según lo mencionado en los criterios, no lo cumple. 
    \item Como se basa en Unittest, tiene una salida casi idéntica al mismo, por
    lo que de igual manera no tiene una salida amigable. 
    \item Tiene una mala documentación de cara a gente novata, pocos ejemplos
    prácticos o supuestos reales, y una muy mala organización.
    \item No tiene para nada un inicio sencillo, el cual se complica por su
    documentación tan mal organizada y escasa. De hecho de sus primeros
    comentarios en la documentación es que si eres un usuario novato consideres
    usar pytest en su lugar.
    \item[\xcmark]
    \href{https://docs.nose2.io/en/latest/plugins/layers.html?highlight=fixtures#organizing-test-fixtures-into-layers}{Permite
    el uso de fixtures} mediante lo que ellos llaman plug-ins. Por lo que si
    cumple este requisito. 
\end{todolist}

\subsubsection{Green}

\begin{todolist}
    \item[\xcmark] Cumple, aunque con un par de inconvenientes. Tiene +160 \emph{issues}
    cerrados y solo 2 abiertos, de los cuales, ambos atendidos, repitiendo la
    misma fórmula para los PR así que por esa parte si cumple. Sin embargo,
    lleva sin recibir commits desde julio de 2021, es decir 6 meses desde el
    último commit en el momento en el que se hace el análisis, lo cual, no está
    del todo mal, pero además de eso el proyecto está mantenido por una única
    persona, cosa que no inspira mucha confianza. A pesar de estos detalles, como
    dije al principio, cumple el requisito.
    \item[\xcmark] Se basa en unittest, sin embargo, a diferencia del anterior, este
    trae añadidos como colorear y mostrar de una forma más clara y visible la
    salida de los informes, por lo que cubre el requisito a pesar de basarse en
    unittest.
    \item Tiene la peor documentación de todas las alternativas vistas, no
    dispone de web dedicada para la documentación, sino que utiliza el README
    del repositorio para ello, siendo bastante escueta, mala y sin ejemplos.
    Tampoco usa la Wiki del propio repositorio, en su lugar te venden
    \href{https://www.udemy.com/course/python-testing-with-green/}{un curso de
    Udemy}, por lo que no cumple para nada este requisito.
    \item Debido a la mala documentación y poca claridad de la misma, tiene unos
    primeros pasos bastante complicados, así que tampoco cumple este requisito.
    \item Al igual que en el requisito anterior, debido a la mala documentación
    que tienen, ha sido difícil averiguar si es compatible con fixtures,
    pero finalmente se ha llegado a la conclusión de que no lo es, ya que no
    aparece nada al respecto en su documentación visible gratuita (desconozco si
    en su curso de pago los mencionarán), y además un usuario en un \emph{issue}
    \href{https://github.com/CleanCut/green/issues/60#issuecomment-112241525}{menciona
    que con fixtures sería fácil de hacer}, de lo que deduzco que no los
    soportan. Por lo que no cumple este requisito
\end{todolist}


\subsubsection{Mamba}

\begin{todolist}
    \item  No cumple el tener una comunidad activa, de hecho a pesar de no
    serlo, parece un proyecto abandonado. No tiene commits desde noviembre de
    2020, y tiene una gran cantidad de \emph{PR} e \emph{issues} abiertos o
    pendientes en comparación con los cerrados.
    \item A pesar de expresar de una forma adecuada el lugar en el que falla, y
    el tiempo de ejecución, no da una salida muy amigable o ``user friendly'', sobre
    todo en comparación con el resto de alternativas, por lo que no cumple este
    requisito.
    \item Tiene una documentación muy escasa y pobre. Sí que tiene un par de
    ejemplos concretos de como empezar, pero poco más. Por lo que no cumple este
    requisito.
    \item[\xcmark] A pesar de tener una documentación mala, los primeros pasos y
    el cómo crear un test simple si está documentado de manera aceptable, por lo
    que el cómo empezar no es excesivamente complicado, cumpliendo este
    requisito. 
    \item No es compatible con el uso de fixtures, ya que en su documentación
    oficial no hay ni mención sobre como implementarlos. Por lo que no cumple
    este requisito.
\end{todolist}



Tras analizar los posibles test runners encontrados, vemos que pocos de ellos
cumplen varios de los requisitos establecidos, siendo Pytest el que más cumple
de todos, por lo que ha sido el elegido finalmente a pesar de no cumplir el
100\% de los requisitos.


\section{Temporización}

\section{Seguimiento del desarrollo}
