\chapter{Implementación}

La implementación del software se ha dividido en hitos. Estos han sido definidos en GitHub
y cada uno de ellos contiene un grupo de \textit{issues} que se corresponden con las distintas
mejoras que se han ido incorporando al software a lo largo de su desarrollo.\\

\section{Selección del lenguaje de programación}

Dentro del proyecto se ven involucrados más de un dispositivo, ya que coexisten
dispositivo móvil y receptor. Ambos podrían llegar a programarse usando el mismo
lenguaje, lo cual sería ventajoso, puesto que, se unificaría el \emph{toolchain} e
incluso se podría ahorrar código. Pero dado que existen requisitos o
dependencias más prioritarias (las cuales serán detalladas en los requisitos de
búsqueda y selección), la elección se dejará a juicio del proceso de selección.

\subsection{Dispositivo móvil}

En este aspecto no tenemos mucho donde elegir. Si hablamos de dispositivos
móviles estamos hablando de \emph{Android} e \emph{iOS}. Y puesto que ya
descartamos \emph{iOS} en la introducción del problema, por ser un sistema
operativo más cerrado y buscar centrarnos en un sistema operativo abierto, el
sistema operativo móvil sobre el que se centrará el proyecto será \emph{Android}.

Si queremos programar en \emph{Android} primero debemos decidir si queremos hacer una
aplicación nativa, híbrida o generada \cite{nativo-hibrido-generado}. Las
aplicaciones nativas se programan en lenguajes con soporte nativo por el sistema
operativo, por lo que tienen un aspecto más similar al propio sistema operativo,
y un mejor rendimiento, sin embargo, requieren de más esfuerzo de desarrollo. Las
aplicaciones híbridas son aplicaciones web que se ejecutan en un navegador web
simulando ser una aplicación nativa, por lo que la interfaz no suele tener un
aspecto tan uniforme al sistema y el rendimiento suele ser notablemente
inferior. Y las generadas son aplicaciones que se programan en un lenguaje no
nativo, pero que mediante un \emph{framework} son traducidas a lenguajes nativos, sin
embargo, al depender de traducción el rendimiento no llega a ser el mismo que el
de una aplicación nativa.

A la hora de elegir qué usar para este proyecto entre los tres tipos de
aplicaciones mencionados, la interfaz gráfica no es algo a tener en cuenta, ya
que, no vamos a hacer una aplicación como tal, sino un servicio para que otras
aplicaciones implementen. Sin embargo, sí que tenemos que tener en cuenta el
rendimiento así como el acceso y manejo con el sistema operativo o con
componentes hardware de más bajo nivel, como es el \emph{NFC}. Para estos casos las
aplicaciones nativas son una mejor elección, ya que, al existir menos capas entre
la aplicación y el sistema operativo, obtienen un mejor rendimiento, y facilitan
la comunicación con componentes hardware de bajo nivel. Lo cual nos lleva a
optar por una aplicación nativa.

Para programar una aplicación nativa existen varios lenguajes que lo permiten
\cite{lenguajes-nativos}, sin embargo, dado que vamos a requerir de comunicación
con elementos hardware como es el lector \emph{NFC}, vamos a optar por lenguajes
que tengan una documentación extensa y detallada en cuanto a desarrollo móvil y
comunicación con \emph{Android}, reduciendo estos lenguajes nativos a \emph{Java} y
\emph{Kotlin}, puesto que, son para los que existe más bibliografía y ejemplos
relacionados con este sistema operativo móvil.

Finalmente, para optar por uno u otro, \emph{Google} lo tiene claro, ya que desde 2019
están enfocados en \emph{Kotlin} y buscan que sea el principal lenguaje de \emph{Android}
\cite{prioridad-kotlin}. De hecho como ellos mismos explicaron, desde 2019 cada
vez que desarrollan una \emph{API} o contenido nuevo lo hacen teniendo en mente
el uso con \emph{Kotlin}, por lo que este lenguaje será el más indicado si lo que
pretendemos es usar alguna \emph{API} que se comunique con el \emph{NFC} del
sistema \emph{Android}.

Así que, teniendo todo esto en cuenta, el lenguaje utilizado para el
dispositivo móvil será \textbf{\emph{Kotlin}}

\subsection{Dispositivo receptor}

Este dispositivo utilizará \emph{Linux}, ya que es el sistema operativo que está
presente en casi la totalidad de placas de desarrollo y mini PC que lo cual
facilitará enormemente el desarrollo. 

\subsubsection{Criterios de búsqueda}

Dado que correr \emph{Linux} no es algo limitante a la hora de elegir lenguaje de
programación, el principal factor que fijará el criterio de búsqueda es la
compatibilidad con librerías NFC actualizadas.

\begin{itemize}
    \item Debe de ser compatible con librerías NFC.
    \item Debe ser un lenguaje tocado previamente en alguno de los años del grado.
\end{itemize}

\subsubsection{Criterios de selección}

Las propuestas encontradas bajo los criterios de búsqueda serán evaluadas bajo
los siguientes criterios de selección:

\begin{itemize}
    \item La librería debe estar mantenida o recibir actualizaciones cada poco tiempo.
    \item Debe de tener documentación extensa en la que se valorarán los ejemplos ya hechos.
    \item La instalación de la misma debe de ser lo más simple posible.
    \item Se valorará muy positivamente la facilidad o simplicidad del lenguaje a la hora de manejar los datos a enviar o recibir con el NFC.
\end{itemize}

\subsubsection{Opciones encontradas según los criterios de búsqueda establecidos}

\begin{itemize}
    \item Java: \href{https://github.com/grundid/nfctools}{nfctools}. Si bien al
    hablar de Java esperaba que existiese una gran cantidad de librerías, no han
    sido tantas las encontradas. Siendo la mayoría librerías comerciales para
    lectores NFC concretos.
    \item Python: \href{https://github.com/nfcpy/nfcpy}{nfcpy}. Al igual que con
    Java, la búsqueda ha devuelto menos librerías de las esperadas. Aunque en
    este caso casi todas las entradas y proyectos existentes con NFC en Python
    apuntaban a esta librería.
    \item C: \href{https://github.com/nfc-tools/libnfc}{libnfc}. 
    \item C++: \href{https://github.com/liblogicalaccess/liblogicalaccess}{LibLogicalAccess}.
\end{itemize}

\subsubsection{Proceso de selección y estudio}

\subsubsection{Java: nfctools}

\begin{todolist}
    \item El repositorio de \href{https://github.com/grundid/nfctools}{GitHub}
    no ha recibido actualización desde 2014. Si bien algún colaborador ha
    seguido trabajando en su propia rama del proyecto algún tiempo, los cambios
    ni siquiera han sido fusionados con \emph{master} e incluso esa rama más
    actualizada lleva 3 años sin un \emph{commit}. Por lo que este requisito no lo
    cumple dicha librería, y, por tanto, tampoco Java.
    \item Esta librería no tiene documentación más que el
    \href{https://github.com/grundid/nfctools/blob/master/README}{README de
    GitHub} y la \href{https://www.grundid.de/nfc/index.html}{web del proyecto},
    la cual hace más de portal inicial que de documentación. Por lo que no
    cumple este requisito.
    \item Al no tener documentación hace que el proceso de instalación requiera
    de intentos a ciegas, investigación y avances según prueba y error. Por lo
    que no cumple este requisito.
    \item[\xcmark] Java es un lenguaje con el que se puede trabajar y manejar
    los datos de forma medianamente fácil, ya que dispone de clases hechas que
    automatizan y facilitan parte del proceso. Aunque igualmente si se compara
    con otros lenguajes habrá alternativas que simplifiquen aún más el proceso.
    Sin embargo, considero este requisito como válido.
\end{todolist}

\subsubsection{Python: nfcpy}

\begin{todolist}
    \item [\xcmark] El último \emph{commit} en el repositorio de
    \href{https://github.com/nfcpy/nfcpy}{GitHub} fue en marzo de 2022 cuando
    lanzó el último \emph{release}. Mientras que la última actividad del principal
    contribuidor de la organización que lleva el repositorio fue a finales de
    agosto contestando issues; por lo que podemos decir que aunque no de manera
    continuada o diaria, la librería sigue recibiendo soporte y está activa, por
    lo que cumple este requisito.
    \item [\xcmark] Posee
    \href{https://nfcpy.readthedocs.io/en/latest/index.html}{readthedocs} en
    donde se detallan los primeros pasos, instalación, ejemplos, etc. Es
    bastante completa y concisa por lo que también satisface con creces este
    requisito.
    \item [\xcmark] Para instalarlo basta con emplear pip, ya que está publicado en
    \href{https://pypi.org/project/nfcpy/}{PyPi}. Por lo que se instala de una
    forma rápida y sencilla.
    \item[\xcmark] Python es uno de los lenguajes más simples y sencillos de
    usar, especialmente como se está viendo últimamente para el manejo de datos,
    por lo que este requisito lo cumple de manera sobrada.
\end{todolist}

\subsubsection{C: libnfc}

\begin{todolist}
    \item [\xcmark] El último \emph{commit} del repositorio de
    \href{https://github.com/nfc-tools/libnfc}{GitHub} fue en febrero de 2023,
    por lo que podríamos decir que si mantiene el soporte, sin embargo,
    revisando los \emph{issues} y los \emph{pull request} están en su mayoría
    desatendidos desde hace años. Y de hecho los \emph{commits} de los últimos años son
    simplemente de fusionar algunos PR aceptados, por lo que da la sensación que
    el soporte que recibe es el de comprobar cada ciertos meses los \emph{pull
    request} activos y aprobarlos si son cambios que no requieren mucha
    intervención. Sin embargo, aún sabiendo esto, interpretaré como que recibe
    soporte, ya que en cierto modo es así.
    \item Posee cierta documentación muy básica en el README del repositorio, y
    adicionalmente tienen \href{nfc-tools.org}{una web} en la que se supone que
    tienen una comunidad, una wiki detallada, etc. Pero esta web no responde en
    el momento de hacer el estudio, por lo que se entiende que la web ha sido
    abandonada, por tanto, no cumple este requisito.
    \item Para instalar la librería primero hay que clonar el repositorio,
    ejecutar el \emph{script} de configuración y una vez generados los archivos
    de configuración, compilar con \emph{make}, para después finalmente ejecutar otro
    \emph{make install}. Una vez completado el \emph{makefile} hay que copiar los binarios
    resultantes a cierta ruta y configurar el \emph{path}. Aunque son unos pasos
    no muy complicados que cualquier estudiante del grado podría superar, no es
    un proceso rápido o sencillo si lo comparamos con el de la librería de
    Python la cual se instalaba automáticamente con pip. Por tanto, considero que
    no supera este requisito.
    \item C no es un lenguaje especialmente simple a la hora de tratar con
    datos, pues no dispone de una STL como es el caso de C++ en el que se
    simplifique el trabajo con vectores etc. Habría que hacerlo manualmente con
    malloc, calloc, free, etc. Por lo que se aleja mucho de la simplicidad o
    facilidad para el manejo. Debido a esto, no supera el requisito.
\end{todolist}

\subsubsection{C++: LibLogicalAccess}

\begin{todolist}
    \item [\xcmark] El último \emph{commit} del repositorio de
    \href{https://github.com/liblogicalaccess/liblogicalaccess}{GitHub} fue en
    marzo de 2023, y los \emph{issues} y \emph{pull request} se mantienen todos
    atendidos, por lo que denota que se mantiene activo y con soporte del
    desarrollador. Por tanto, cumple este requisito.
    \item [\xcmark] Utiliza la wiki de GitHub a modo de documentación, teniendo
    documentado el proceso de instalación y algunos ejemplos en hardware
    concreto. Además, posee una \href{https://liblogicalaccess.com/}{web oficial}
    en la que tienen publicado el Doxygen. Podemos decir que cumple dicho requisito,
    aunque comparando con algunas otras alternativas vistas, no es la más
    completa.
    \item Para instalar la librería hay que emplear un software llamado
    \href{https://conan.io/}{conan}, el cual es un gestor de paquetes para C y
    C++ tras clonar y compilar el repositorio con conan. Adicionalmente, este
    software se instala a través de pip, por lo que habría que instalar también
    pip, ya que con una instalación de gcc no se instala. Por tanto, no considero
    que sea una instalación necesariamente simple, por lo que no cumple este
    requisito.
    \item [\xcmark] A pesar de que C++ tiene la STL la cual simplifica mucho el manejo de
    datos con las clases ya hechas y la automatización de la memoria dinámica
    que se hace dentro de la misma, el manejo de datos no es la tarea más simple
    dentro de C++. Sin embargo, aunque sea por los pelos o ``cogido con pinzas'',
    daré como cumplido este requisito.
\end{todolist}

\subsubsection{Conclusión}

Tras valorar mediante los criterios de selección los lenguajes principales
encontrados según los criterios de búsqueda, podemos ver que el lenguaje que
satisface todos los criterios es \textbf{Python con la librería nfcpy}. Por lo que este
será el lenguaje empleado en la parte del receptor.