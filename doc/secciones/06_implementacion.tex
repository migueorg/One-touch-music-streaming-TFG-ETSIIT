\chapter{Implementación}

La implementación del software se ha dividido en hitos. Estos han sido definidos en GitHub
y cada uno de ellos contiene un grupo de \textit{issues} que se corresponden con las distintas
mejoras que se han ido incorporando al software a lo largo de su desarrollo.\\

\section{Selección del lenguaje de programación}

Dentro del proyecto se ven involucrados más de un dispositivo, ya que coexisten
dispositivo móvil y receptor, los cuales no se podrán programar usando el mismo
lenguaje, o no tendrá por qué ser lo más adecuado.

\subsection{Dispositivo móvil}

En este aspecto no tenemos mucho donde elegir. Si hablamos de dispositivos
móviles estamos hablando de \emph{Android} e \emph{iOS}. Y puesto que ya
descartamos \emph{iOS} en la introducción del problema, por ser un sistema
operativo más cerrado y buscar centrarnos en un sistema operativo abierto, el
sistema operativo móvil sobre el que se centrará el proyecto será \emph{Android}.

Si queremos programar para \emph{Android}, las opciones son más bien escasas.
Teniendo como opciones \emph{Kotlin} y \emph{Java}. Por lo que siguiendo las
recomendaciones de \emph{Android}, el mejor camino para programar aplicaciones
nativas más seguras, concisas y estructuradas es \textbf{\emph{Kotlin}}. El cual está
pensado para sustituir a \emph{Java} como lenguaje en aplicaciones
\emph{Android} nativas.