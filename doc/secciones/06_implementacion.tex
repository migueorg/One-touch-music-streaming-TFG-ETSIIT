\chapter{Implementación}

La implementación del software se ha dividido en hitos. Estos han sido definidos en GitHub
y cada uno de ellos contiene un grupo de \textit{issues} que se corresponden con las distintas
mejoras que se han ido incorporando al software a lo largo de su desarrollo.\\

\section{Selección del lenguaje de programación}

Dentro del proyecto se ven involucrados más de un dispositivo, ya que coexisten
dispositivo móvil y receptor, los cuales no se podrán programar usando el mismo
lenguaje, o no tendrá por qué ser lo más adecuado.

\subsection{Dispositivo móvil}

En este aspecto no tenemos mucho donde elegir. Si hablamos de dispositivos
móviles estamos hablando de \emph{Android} e \emph{iOS}. Y puesto que ya
descartamos \emph{iOS} en la introducción del problema, por ser un sistema
operativo más cerrado y buscar centrarnos en un sistema operativo abierto, el
sistema operativo móvil sobre el que se centrará el proyecto será \emph{Android}.

Si queremos programar para \emph{Android}, las opciones son más bien escasas.
Teniendo como opciones \emph{Kotlin} y \emph{Java}. Por lo que siguiendo las
recomendaciones de \emph{Android}, el mejor camino para programar aplicaciones
nativas más seguras, concisas y estructuradas es \textbf{\emph{Kotlin}}. El cual está
pensado para sustituir a \emph{Java} como lenguaje en aplicaciones
\emph{Android} nativas.

\subsection{Dispositivo receptor}

Este dispositivo utilizará \emph{Linux}, ya que es el sistema operativo que está
presente en casi la totalidad de placas de desarrollo y mini PC que lo cual
facilitará enormemente el desarrollo. 

\subsubsection{Criterios de búsqueda}

Dado que el uso de \emph{Linux} no es algo limitante a la hora de elegir lenguaje de
programación, el principal factor que fijará el criterio de búsqueda es la
compatibilidad con librerías NFC actualizadas.

\begin{itemize}
    \item Debe de ser compatible con librerías NFC.
    \item Debe ser un lenguaje usado previamente en alguno de los años del grado.
\end{itemize}

\subsubsection{Criterios de selección}

Las propuestas encontradas bajo los criterios de búsqueda serán evaluadas bajo
los siguientes criterios de selección:

\begin{itemize}
    \item La librería debe estar mantenida o recibir actualizaciones cada poco tiempo.
    \item Debe de tener documentación extensa en la que se valorarán los ejemplos ya hechos.
    \item La instalación de la misma debe de ser lo más simple posible.
    \item Se valorará muy positivamente la facilidad o simplicidad del lenguaje a la hora de manejar los datos a enviar o recibir con el NFC.
\end{itemize}

\subsubsection{Opciones encontradas según los criterios de búsqueda establecidos}

\begin{itemize}
    \item Java: \href{https://github.com/grundid/nfctools}{nfctools}. Si bien al
    hablar de Java esperaba que existiese una gran cantidad de librerías, no han
    sido tantas las encontradas. Siendo la mayoría librerías comerciales para
    lectores NFC concretos.
    \item Python: \href{https://github.com/nfcpy/nfcpy}{nfcpy}. Al igual que con
    Java, la búsqueda ha devuelto menos librerías de las esperadas. Aunque en
    este caso casi todas las entradas y proyectos existentes con NFC en Python
    apuntaban a esta librería.
    \item C: \href{https://github.com/nfc-tools/libnfc}{libnfc}. 
    \item C++: \href{https://github.com/liblogicalaccess/liblogicalaccess}{LibLogicalAccess}.
\end{itemize}