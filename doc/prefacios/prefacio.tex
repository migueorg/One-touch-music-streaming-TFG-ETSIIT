\thispagestyle{empty}

\begin{center}
{\large\bfseries Sistema de streaming por contacto usando dispositivos de bajo coste \\ Haciendo el streaming sencillo }\\
\end{center}
\begin{center}
Miguel Ángel Martín Rodríguez\\
\end{center}

%\vspace{0.7cm}

\vspace{0.5cm}
\noindent\textbf{Palabras clave}: \textit{software libre}, \textit{NFC}, \textit{streaming}, \textit{audio}, \textit{kotlin}
\vspace{0.7cm}

\noindent\textbf{Resumen}\\
En los tiempos en los que estamos, cada vez es más normal escuchar música o
podcasts de fondo mientras hacemos otras tareas de nuestro día a día. Escuchamos
contenido mientras trabajamos, cocinamos, limpiamos, vamos en el coche, de
camino al trabajo, de vuelta a casa, entre muchos otros más escenarios. 

A su vez, en el contexto informático en el que estamos, también le hemos dado la
espalda a los cables siempre que nos sea posible. Los auriculares inalámbricos
son un obligatorio en el día a día de muchas personas, facilitando así la
integración de la música en cada vez más contextos diferentes de nuestra vida.
Sin embargo, aún teniendo en cuenta los avances en el campo del \emph{IoT}
\footnote{Del inglés ``Internet Of Things'', es decir, ``Internet de las cosas'', es
la agrupación e interconexión de dispositivos y objetos a través de una red
(bien sea privada o Internet, la red de redes), dónde todos ellos podrían ser
visibles e interaccionar. \cite{IoT}}, no es tan fácil mantener esa sensación de
integración y continuidad respecto a la escucha de audio. Ya que si queremos
seguir escuchando en algún altavoz inteligente la misma canción o podcast que
estamos escuchando en el móvil, tendremos que empezar a interactuar con menús,
o bien dictar comandos (que no siempre funcionan) al asistente virtual; o alguna otra
alternativa compleja o cara.\\

Este proyecto busca simplificar la tarea de envío de audio en streaming desde el
móvil hasta el dispositivo receptor.

\cleardoublepage
% textidote: ignore begin %
\begin{center}
	{\large\bfseries Contact streaming system using low-cost devices}\\
\end{center}
\begin{center}
	Miguel Ángel Martín Rodríguez\\
\end{center}
\vspace{0.5cm}
\noindent\textbf{Keywords}: \textit{open source}, \textit{floss}
\vspace{0.7cm}

\noindent\textbf{Abstract}\\
In this day and age, it is becoming more and more normal to listen to music or
podcasts in the background while we do other tasks in our daily lives. We listen
to while working, cooking, cleaning, driving, on the way to work, on the way
home, among many other scenarios. 

At the same time, in the IT context we are in, we have also turned our backs on
cables whenever it is convenient for us. to wires whenever possible. Wireless
headsets are a must in many people's daily lives, facilitating the integration
of music into more and more integration of music into more and more different
contexts of our lives. However, even taking into account the advances in the
field of \emph{IoT} \footnote{```Internet Of Things'', it is the grouping and
interconnection of devices and objects through a network (either private or the
Internet, the network of networks), where all of them could be visible and
interacting. \cite{IoT}}, it is not so easy to maintain that sense of
integration and continuity with respect to audio listening. Because if we want
to continue listening on a smart speaker to the same song or podcast that we are
listening to on the mobile, we will have to start interacting with menus, or
dictate commands (which do not always work) to the virtual assistant; or some
other complex or expensive alternative.

This project seeks to simplify the task of sending streaming audio from the
mobile to the receiving device.
% textidote: ignore end %
\cleardoublepage

\thispagestyle{empty}

\noindent\rule[-1ex]{\textwidth}{2pt}\\[4.5ex]

D. \textbf{Juan Julián Merelo Guervós}, Profesor del departamento de Arquitectura y Tecnología de Computadores.

\vspace{0.5cm}

\textbf{Informo:}

\vspace{0.5cm}

Que el presente trabajo, titulado \textit{\textbf{Sistema de streaming por contacto usando dispositivos de bajo coste}},
ha sido realizado bajo mi supervisión por \textbf{Miguel Ángel Martín Rodríguez}, y autorizo la defensa de dicho trabajo ante el tribunal
que corresponda.

\vspace{0.5cm}

Y para que conste, expiden y firman el presente informe en Granada a Noviembre de 2023.

\vspace{1cm}

\textbf{El/la director(a)/es: }

\vspace{5cm}

\noindent \textbf{Juan Julián Merelo Guervós}

\chapter*{Agradecimientos}

A mi familia, a mi tutor y a mis amigos.


